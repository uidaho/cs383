\documentclass[14pt, a4paper]{article}

\usepackage{float}
\usepackage{framed}
\usepackage{comment}
\usepackage{enumitem}
\usepackage{listings}
\usepackage{tabto}
\usepackage[a4paper]{geometry}
\usepackage{tikz}
\graphicspath{ {images/}}
\geometry{a4paper,total={210mm,297mm},left=30mm,right=30mm,top=30mm,bottom=30mm}

\begin{document}

\noindent Author: Domn Werner (wern0096) \\
Class: CS383-01 \\
Date: 01/31/2016 \\
Assignment: HW\#2

\tableofcontents

\newpage

\section{File Management}

\subsection{Add New File to Project}

\begin{framed}
	\noindent\textbf{Task Name:} Add New File to Project \\ \\
	\textbf{Task Category:} File Management \\ \\
	\textbf{Actor:} User \\ \\
	\textbf{Summary:} The user performs this task to add a new file to the project. \\ \\
	\textbf{Preconditions:} 
	\begin{enumerate}
		\item User must be registered.
		\item User must be logged in.
		\item User must have a project open.
	\end{enumerate}
	\textbf{Steps:}
	\begin{enumerate}
		\item User clicks \textit{File} in the top menu bar.
		\item System opens a drop-down menu.
		\item User navigates to \textit{Add} -$>$ \textit{New File}.
		\item System opens an \textit{Add New File} dialog window.
		\item User selects the file type and names the file.
		\item User clicks \textit{Add}.
		\item System adds the file to the project.
	\end{enumerate}
	\textbf{Alternatives:} 
	\begin{enumerate}
		\item Step 1: The user right clicks in the project pane and the system continues on to step 2 above.
		\item Step 5: The user clicks \textit{Cancel} and a new file is not added to the project.
	\end{enumerate}
	\textbf{Postconditions:}
	\begin{enumerate}
		\item A new file is added to the project.
		\item The database is updated to reflect the changes.
	\end{enumerate}
	\textbf{Related:} Add Existing File to Project
\end{framed} 

\newpage

\subsection{Add Existing File to Project}

\begin{framed}
	\noindent\textbf{Task Name:} Add Existing File to Project \\ \\
	\textbf{Task Category:} File Management \\ \\
	\textbf{Actor:} User \\ \\
	\textbf{Summary:} The user performs this task to add an existing file to the project. \\ \\
	\textbf{Preconditions:} 
	\begin{enumerate}
		\item User must be registered.
		\item User must be logged in.
		\item User must have a project open.
	\end{enumerate}
	\textbf{Steps:}
	\begin{enumerate}
		\item User clicks \textit{File} in the top menu bar.
		\item System opens a drop-down menu.
		\item User navigates to \textit{Add} -$>$ \textit{Existing File}.
		\item System opens an \textit{Add Existing File} dialog window.
		\item User selects \textit{PC} or \textit{SQuire} or \textit{Github}.
		\item System updates the dialog to reflect the selected source.
		\item User navigates to the file's location and selects it.
		\item User clicks \textit{Add}.
		\item System adds the file to the project.
	\end{enumerate}
	\textbf{Alternatives:} 
	\begin{enumerate}
		\item Step 1: The user right clicks in the project pane and the system continues on to step 2 above.
		\item Step 5-7: The user clicks \textit{Cancel} and a new file is not added to the project.
	\end{enumerate}
	\textbf{Postconditions:}
	\begin{enumerate}
		\item An existing file is added to the project.
		\item The database is updated to reflect the changes.
	\end{enumerate}
	\textbf{Related:} Add New File to Project
\end{framed} 

\newpage

\subsection{Delete File}

\begin{framed}
	\noindent\textbf{Task Name:} Delete File \\ \\
	\textbf{Task Category:} File Management \\ \\
	\textbf{Actor:} User \\ \\
	\textbf{Summary:} The user performs this task to delete a file from the project. \\ \\
	\textbf{Preconditions:} 
	\begin{enumerate}
		\item User must be registered.
		\item User must be logged in.
		\item User must have a project open.
		\item Current project must have at least one file.
	\end{enumerate}
	\textbf{Steps:}
	\begin{enumerate}
		\item User clicks right clicks a file in the project pane.
		\item System opens a drop-down menu.
		\item User navigates to \textit{Delete}.
		\item System opens an \textit{Delete File} dialog window, asking if the user is sure.
		\item User selects \textit{Yes}.
		\item System deletes the file from the project.
	\end{enumerate}
	\textbf{Alternatives:} 
	\begin{enumerate}
		\item Step 5: The user clicks \textit{Cancel} instead and the file is not deleted from the project.
		\item The user selects multiple files before step 1.
	\end{enumerate}
	\textbf{Postconditions:}
	\begin{enumerate}
		\item The file is deleted from the project.
		\item The database is updated to reflect the changes.
	\end{enumerate}
\end{framed} 

\newpage

\subsection{Export Project}

\begin{framed}
	\noindent\textbf{Task Name:} Export Project \\ \\
	\textbf{Task Category:} File Management \\ \\
	\textbf{Actor:} User \\ \\
	\textbf{Summary:} The user performs this task to download a project as a zip file. \\ \\
	\textbf{Preconditions:} 
	\begin{enumerate}
		\item User must be registered.
		\item User must be logged in.
		\item User must have a project open.
		\item User must have download permissions.
	\end{enumerate}
	\textbf{Steps:}
	\begin{enumerate}
		\item User clicks \textit{File} in the top menu bar.
		\item System opens a drop-down menu.
		\item User navigates to \textit{Export} -$>$ \textit{Project}.
		\item System opens an \textit{Export} dialog window.
		\item User navigates to the export location.
		\item User clicks \textit{Export}.
		\item System zips the file and downloads it to the specified location.
	\end{enumerate}
	\textbf{Alternatives:} 
	\begin{enumerate}
		\item Step 1: The user right clicks in the project pane and the system continues on to step 2 above.
		\item Step 5: The user clicks \textit{Cancel} and the project is not exported.
	\end{enumerate}
	\textbf{Related:} Export Files
\end{framed}

\newpage

\subsection{Export Files}

\begin{framed}
	\noindent\textbf{Task Name:} Export Files \\ \\
	\textbf{Task Category:} File Management \\ \\
	\textbf{Actor:} User \\ \\
	\textbf{Summary:} The user performs this task to download a number of files from a project. \\ \\
	\textbf{Preconditions:} 
	\begin{enumerate}
		\item User must be registered.
		\item User must be logged in.
		\item User must have a project open.
		\item Must have at least one file in the project.
		\item User must have download permissions.
	\end{enumerate}
	\textbf{Steps:}
	\begin{enumerate}
		\item User clicks \textit{File} in the top menu bar.
		\item System opens a drop-down menu.
		\item User navigates to \textit{Export} -$>$ \textit{Files}.
		\item System opens an \textit{Export} dialog window showing the project files on the left pane and the export location in the right pane.
		\item User selects a number of files on the left pane.
		\item User navigates to the export location in the right pane.
		\item User clicks \textit{Export}.
		\item System downloads the selected files to the specified location.
	\end{enumerate}
	\textbf{Alternatives:} 
	\begin{enumerate}
		\item Step 1: The user right clicks in the project pane and the system continues on to step 2 above.
		\item Step 5: User selects a folder and all files under that folder are selected.
		\item Step 5-6: The user clicks \textit{Cancel} and the project is not exported.
	\end{enumerate}
	\textbf{Related:} Export Project
\end{framed}

\newpage

\subsection{Open File in New Tab}

\begin{framed}
	\noindent\textbf{Name:} Open File in New Tab \\ \\
	\textbf{Category:} File Management \\ \\
	\textbf{Actor:} User \\ \\
	\textbf{Summary:} Allows users to open a file. \\ \\
	\textbf{Purpose:} Opening files is essential in being able to work on a project. \\ \\
	\textbf{Preconditions:} 
	\begin{enumerate}
		\item User is registered.
		\item User is logged in.
		\item User has a project open.
		\item Current project contains at least one file.
		\item User has read permission.
	\end{enumerate}		
	\textbf{Steps:}
	\begin{enumerate}
		\item User double clicks a file.
		\item The editor opens its contents in a new tab and focuses on it.
	\end{enumerate}	
	\textbf{Alternatives:} Step 1: Instead of double clicking a file, the user right clicks it and navigates to \textit{Open}.
\end{framed}

\newpage

\subsection{Save File}

\begin{framed}
	\noindent\textbf{Name:} Save File \\ \\
	\textbf{Category:} File Management \\ \\
	\textbf{Actor:} User \\ \\
	\textbf{Summary:} Allows users save edited files. \\ \\
	\textbf{Purpose:} Saving files is essential in maintaining the integrity of a project. \\ \\
	\textbf{Preconditions:} 
	\begin{enumerate}
		\item User is registered.
		\item User is logged in.
		\item User has a project open.
		\item Project contains at least one file.
		\item User has read permission.
		\item User has write permission.
		\item At least one file has been edited since the last save.
	\end{enumerate}		
	\textbf{Steps:}
	\begin{enumerate}
		\item User edits one or more files.
		\item System enables a flag signaling that edits have occurred.
		\item User clicks a save button or keybind.
		\item System saves current open file.
	\end{enumerate}	
	\textbf{Alternatives:} 
	\begin{enumerate}
		\item A \textit{Save All} button that saves all open, edited files.
	\end{enumerate}
	\textbf{Postconditions:}
	\begin{enumerate}
		\item The database is update to reflect the save operation.
		\item The save button is grayed out.
		\item The save all button is grayed out if it was clicked.
	\end{enumerate}
\end{framed}

\newpage

\subsection{Close Saved File}

\begin{framed}
	\noindent\textbf{Name:} Close Saved File \\ \\
	\textbf{Category:} File Management \\ \\
	\textbf{Actor:} User \\ \\
	\textbf{Summary:} Allows users to close opened, saved files. \\ \\
	\textbf{Purpose:} Closing files is essential in managing the workspace and increasing productivity. \\ \\
	\textbf{Preconditions:} 
	\begin{enumerate}
		\item User is registered.
		\item User is logged in.
		\item User has a project open.
		\item User has at least one file open.
		\item User has read permission
		\item User has write permission.
		\item Current opened file has no unsaved changes.
	\end{enumerate}		
	\textbf{Steps:}
	\begin{enumerate}
		\item User clicks the 'x' on the edge of an open tab.
		\item System closes current open file.
	\end{enumerate}	
	\textbf{Postconditions:}
	\begin{enumerate}
		\item The opened tab and file are closed.
	\end{enumerate}
	\textbf{Related:} Close Unsaved File
\end{framed}

\newpage

\subsection{Close Unsaved File}

\begin{framed}
	\noindent\textbf{Name:} Close Unsaved File \\ \\
	\textbf{Category:} File Management \\ \\
	\textbf{Actor:} User \\ \\
	\textbf{Summary:} Allows users to close opened, unsaved files. \\ \\
	\textbf{Purpose:} Closing unsaved files is a possibly risky procedure. It is important that the use case reduces the chance of the user losing their work. \\ \\
	\textbf{Preconditions:} 
	\begin{enumerate}
		\item User is registered.
		\item User is logged in.
		\item User has a project open.
		\item User has at least one file open.
		\item User has read permission
		\item User has write permission.
		\item Current opened file has unsaved changes.
	\end{enumerate}		
	\textbf{Steps:}
	\begin{enumerate}
		\item User clicks the 'x' on the edge of an open tab.
		\item System presents a dialog telling the user that there are unsaved changes to the file.
		\item System asks the use if he wants to \textit{Save Changes and Don't Close}, \textit{Save and Close}, \textit{Close and Don't Save}, or \textit{Cancel}.
	\end{enumerate}	
	\textbf{Alternatives:}
	\begin{enumerate}
		\item If the user selects \textit{Save Changes}, the file is saved as per the Save File use-case.
		\item If the user selects \textit{Save and Close}, the file is saved as per the Save File use-case and closed as per the Close Saved File use-case.
		\item If the user selects \textit{Close and Don't Save}, the file is closed.
		\item If the user selects \textit{Cancel}, nothing happens.
	\end{enumerate}
	\textbf{Postconditions:}
	\begin{enumerate}
		\item The database is updated if the user chose to save the file.
	\end{enumerate}
	\textbf{Related:} 
	\begin{enumerate}
		\item Close Saved File
		\item Save File
	\end{enumerate}
\end{framed}

\subsection{Use-Case Diagram}

\hspace*{-1cm}\includegraphics[scale=0.45]{fileManagementDiagram}


\newpage	
		
\section{File Editing}

\subsection{View Line Numbers}

\begin{framed}
	\noindent\textbf{Name:} View Line Numbers \\ \\
	\textbf{Category:} File Editing \\ \\
	\textbf{Summary:} Allows the user to view line numbers to the left of the document. \\ \\
	\textbf{Purpose:} Makes it easier to communicate position in code. It is also a useful metric to have.\\ \\
	\textbf{Preconditions:} 
	\begin{enumerate}
		\item Must be registered.
		\item Must be logged in.
		\item User has view permission.
		\item A file is open.
	\end{enumerate}
	\textbf{Steps:}
	\begin{enumerate}
		\item User selects the \textit{View} menu option.
		\item System displays a drop-down with various options.
		\item User selects the \textit{View Line Numbers} option.
		\item System displays line numbers to the left of the document.
	\end{enumerate}
	\textbf{Related:}
	\begin{enumerate}
		\item View References
		\item View Dates
		\item View Authors
	\end{enumerate}
\end{framed}

\newpage

\subsection{View References}

\begin{framed}
	\noindent\textbf{Name:} View References \\ \\
	\textbf{Category:} File Editing \\ \\
	\textbf{Summary:} Allows the user to view the number of references to a given function. \\ \\
	\textbf{Purpose:} It is useful to know the number of references to a given function for optimization and debugging purposes. \\ \\
	\textbf{Preconditions:} 
	\begin{enumerate}
		\item Must be registered.
		\item Must be logged in.
		\item User has view permission.
		\item A \textbf{code} file is open.
	\end{enumerate}
	\textbf{Steps:}
	\begin{enumerate}
		\item User selects the \textit{View} menu option.
		\item System displays a drop-down with various options.
		\item User selects the \textit{View References} option.
		\item System displays the number of references above each method declaration.
	\end{enumerate}
	\textbf{Related:}
	\begin{enumerate}
		\item View Line Numbers
		\item View Dates
		\item View Authors
	\end{enumerate}
\end{framed}

\newpage

\subsection{View Dates}

\begin{framed}
	\noindent\textbf{Name:} View Dates \\ \\
	\textbf{Category:} File Editing \\ \\
	\textbf{Summary:} Allows the user to view the last date that each line of a document was edited. \\ \\
	\textbf{Purpose:} This provides a useful metric for how up-to-date parts of the document are. \\ \\
	\textbf{Preconditions:} 
	\begin{enumerate}
		\item Must be registered.
		\item Must be logged in.
		\item User has view permission.
		\item A file is open.
	\end{enumerate}
	\textbf{Steps:}
	\begin{enumerate}
		\item User selects the \textit{View} menu option.
		\item System displays a drop-down with various options.
		\item User selects the \textit{View Dates} option.
		\item System displays the last date that each line of a document was edited.
	\end{enumerate}
	\textbf{Related:}
	\begin{enumerate}
		\item View Line Numbers
		\item View References
		\item View Authors
	\end{enumerate}
\end{framed}

\newpage

\subsection{View Authors}

\begin{framed}
	\noindent\textbf{Name:} View Authors \\ \\
	\textbf{Category:} File Editing \\ \\
	\textbf{Summary:} Allows the user to view the last author that edited each of line of the document. \\ \\
	\textbf{Purpose:} This is an accountability tool allowing other users to identify who is responsible for a change to a document. \\ \\
	\textbf{Preconditions:} 
	\begin{enumerate}
		\item Must be registered.
		\item Must be logged in.
		\item User has read permission.
		\item A file is open.
	\end{enumerate}
	\textbf{Steps:}
	\begin{enumerate}
		\item User selects the \textit{View} menu option.
		\item System displays a drop-down with various options.
		\item User selects the \textit{View Authors} option.
		\item System displays the name of the last editor of each line of the document.
	\end{enumerate}
	\textbf{Related:}
	\begin{enumerate}
		\item View Line Numbers
		\item View References
		\item View Dates
	\end{enumerate}
\end{framed}

\newpage

\subsection{Format Document}

\begin{framed}
	\noindent\textbf{Name:} Format Document \\ \\
	\textbf{Category:} File Editing \\ \\
	\textbf{Summary:} Allows the user to format the document to a specified format \\ \\
	\textbf{Purpose:} An easy tool for making sweeping changes to a large part of a document. \\ \\
	\textbf{Preconditions:} 
	\begin{enumerate}
		\item Must be registered.
		\item Must be logged in.
		\item User has read/write permission.
		\item A file is open.
		\item The document has formatting options set.
	\end{enumerate}
	\textbf{Steps:}
	\begin{enumerate}
		\item User selects the \textit{Edit} menu option.
		\item System displays a drop-down with various options.
		\item User selects the \textit{Format Document} option.
		\item System formats the current document to the formatting settings currently set.
	\end{enumerate}
	\textbf{Alternatives:}
	\begin{enumerate}
		\item If no formatting settings are currently set, display a dialog box after step 3 and give the option for the user to do so now.
	\end{enumerate}
	\textbf{Related:}
	\begin{enumerate}
		\item Find/Replace
		\item Comment Section
	\end{enumerate}
\end{framed}

\newpage

\subsection{Find/Replace}

\begin{framed}
	\noindent\textbf{Name:} Find/Replace \\ \\
	\textbf{Category:} File Editing \\ \\
	\textbf{Summary:} Allows the user to find and/or replace phrases. \\ \\
	\textbf{Purpose:} This is a powerful tool that allows a user to make safer, quicker, and more efficient changes to a document. \\ \\
	\textbf{Preconditions:} 
	\begin{enumerate}
		\item Must be registered.
		\item Must be logged in.
		\item User has read/write permission.
		\item A file is open.
	\end{enumerate}
	\textbf{Steps:}
	\begin{enumerate}
		\item User selects the \textit{Edit} menu option.
		\item System displays a drop-down with various options.
		\item User selects the \textit{Find/Replace} option.
		\item System displays a small form in an unobtrusive location.
		\item User enter the phrase to find and selects find.
		\item System highlights and focuses on the first occurrence of the phrase and all highlights all other occurrences.
	\end{enumerate}
	\textbf{Alternatives:}
	\begin{enumerate}
		\item User selects option to replace in step 5 and enters a phrase with which to replace the found occurrences of the searched phrase. The system replaces each occurrence.
	\end{enumerate}
	\textbf{Related:}
	\begin{enumerate}
		\item Format Document
		\item Find/Replace
	\end{enumerate}
\end{framed}

\newpage

\subsection{Comment Section}

\begin{framed}
	\noindent\textbf{Name:} Comment Section \\ \\
	\textbf{Category:} File Editing \\ \\
	\textbf{Summary:} Allows the user to comment out a part of a document. \\ \\
	\textbf{Purpose:} A useful and quick way to disable a large part of a document. \\ \\
	\textbf{Preconditions:} 
	\begin{enumerate}
		\item Must be registered.
		\item Must be logged in.
		\item A file is open.
		\item User has read/write permission.
		\item Current open document supports commenting.
	\end{enumerate}
	\textbf{Steps:}
	\begin{enumerate}
		\item User selects the \textit{Edit} menu option.
		\item System displays a drop-down with various options.
		\item User selects the \textit{Comment Section} option.
		\item System comments the selected area.
	\end{enumerate}
	\textbf{Alternatives:}
	\begin{enumerate}
		\item If document does not support commenting, display a dialog box telling the user.
	\end{enumerate}
	\textbf{Related:}
	\begin{enumerate}
		\item Format Document
		\item Find/Replace
	\end{enumerate}
\end{framed}

\newpage

\subsection{Display Typing User}

\begin{framed}
	\noindent\textbf{Name:} Display Typing User \\ \\
	\textbf{Category:} File Editing \\ \\
	\textbf{Summary:} As the user types, the system displays their name, their typing, and their caret, in a different color, to other users. \\ \\
	\textbf{Purpose:} Differentiate who is typing what. \\ \\
	\textbf{Preconditions:} 
	\begin{enumerate}
		\item Must be registered.
		\item Must be logged in.
		\item User has read/write permission.
		\item A file is open.
		\item Other users have the same document open.
	\end{enumerate}
	\textbf{Steps:}
	\begin{enumerate}
		\item User begins typing.
		\item System displays the user's typing, the user's name, and the user's caret, in a different color, to other users.
	\end{enumerate}
\end{framed}

\newpage

\subsection{Display Syntax Errors}

\begin{framed}
	\noindent\textbf{Name:} Display Syntax Errors \\ \\
	\textbf{Category:} File Editing \\ \\
	\textbf{Summary:} As the user types code, the editor will underline syntax errors with a red line. \\ \\
	\textbf{Purpose:} Aids the user is writing correct code. \\ \\
	\textbf{Preconditions:} 
	\begin{enumerate}
		\item Must be registered.
		\item Must be logged in.
		\item User has read/write permission.
		\item A supported code file is open.
	\end{enumerate}
	\textbf{Steps:}
	\begin{enumerate}
		\item User begins typing.
		\item System displays any syntax errors as a red underline under the incorrect section.
	\end{enumerate}
	\textbf{Related:} Display Syntax Highlighting
\end{framed}

\newpage

\subsection{Display Syntax Highlighting}

\begin{framed}
	\noindent\textbf{Name:} Display Syntax Highlighting \\ \\
	\textbf{Category:} File Editing \\ \\
	\textbf{Summary:} As the user types code, the editor will change font color for different code structures and keywords. \\ \\
	\textbf{Purpose:} Aids the user is writing code and identifying key code parts. \\ \\
	\textbf{Preconditions:} 
	\begin{enumerate}
		\item Must be registered.
		\item Must be logged in.
		\item User has read/write permission.
		\item A supported code file is open.
	\end{enumerate}
	\textbf{Steps:}
	\begin{enumerate}
		\item User begins typing.
		\item System automatically colors special code structures and keywords.
	\end{enumerate}
	\textbf{Related:} Display Syntax Errors
\end{framed}

\end{document}