\documentclass[14pt, a4paper]{article}

\usepackage{float}
\usepackage{framed}
\usepackage{comment}
\usepackage{enumitem}
\usepackage{listings}
\usepackage{tabto}
\usepackage[a4paper]{geometry}
\usepackage{tikz}
\graphicspath{ {images/}}
\geometry{a4paper,total={210mm,297mm},left=30mm,right=30mm,top=30mm,bottom=30mm}

\begin{document}

\noindent Author: Domn Werner (wern0096) \\
Date: 01/31/2016 \\
Class: CS383-01

\tableofcontents

\newpage

\section{Project Management}

\subsection{Create New Project}

\begin{framed}
	\noindent\textbf{Name:} Create New Project \\ \\
	\textbf{Category:} Project Management \\ \\
	\textbf{Summary:} The user performs this task to create a new project. \\ \\
	\textbf{Preconditions:} 
	\begin{enumerate}
		\item User must be registered.
		\item User must be logged in.
	\end{enumerate}
	\textbf{Steps:}
	\begin{enumerate}
		\item User clicks \textit{File} in the top menu bar.
		\item System opens a drop-down menu.
		\item User selects \textit{New} -$>$ \textit{Project}.
		\item System opens a \textit{New Project} dialog window.
		\item User fills out the name of the project, selects a language, enters a description, and selects the \textit{Private} or \textit{Public} radio buttons.
		\item User clicks the \textit{Create} button.
		\item System creates a new project and displays it.
	\end{enumerate}
	\textbf{Alternatives:} 
	\begin{enumerate}
		\item Step 1: The user right clicks in the project pane and the system continues to step 2.
		\item Step 5: The user clicks \textit{Cancel} and a new project is not created.
	\end{enumerate}
	\textbf{Postconditions:}
	\begin{enumerate}
		\item A new project with the selected settings is created.
		\item The database is updated to reflect the new project and its data.
	\end{enumerate}
\end{framed} 

\newpage

\subsection{Add New File to Project}

\begin{framed}
	\noindent\textbf{Task Name:} Add New File to Project \\ \\
	\textbf{Task Category:} Project Management \\ \\
	\textbf{Summary:} The user performs this task to add a new file to the project. \\ \\
	\textbf{Preconditions:} 
	\begin{enumerate}
		\item User must be registered.
		\item User must be logged in.
		\item User must have a project open.
	\end{enumerate}
	\textbf{Steps:}
	\begin{enumerate}
		\item User clicks \textit{File} in the top menu bar.
		\item System opens a drop-down menu.
		\item User navigates to \textit{Add} -$>$ \textit{New File}.
		\item System opens an \textit{Add New File} dialog window.
		\item User selects the file type and names the file.
		\item User clicks \textit{Add}.
		\item System adds the file to the project.
	\end{enumerate}
	\textbf{Alternatives:} 
	\begin{enumerate}
		\item Step 1: The user right clicks in the project pane and the system continues on to step 2 above.
		\item Step 5: The user clicks \textit{Cancel} and a new file is not added to the project.
	\end{enumerate}
	\textbf{Postconditions:}
	\begin{enumerate}
		\item A new file is added to the project.
		\item The database is updated to reflect the changes.
	\end{enumerate}
	\textbf{Related:} Add Existing File to Project
\end{framed} 

\newpage

\subsection{Add Existing File to Project}

\begin{framed}
	\noindent\textbf{Task Name:} Add Existing File to Project \\ \\
	\textbf{Task Category:} Project Management \\ \\
	\textbf{Summary:} The user performs this task to add an existing file to the project. \\ \\
	\textbf{Preconditions:} 
	\begin{enumerate}
		\item User must be registered.
		\item User must be logged in.
		\item User must have a project open.
	\end{enumerate}
	\textbf{Steps:}
	\begin{enumerate}
		\item User clicks \textit{File} in the top menu bar.
		\item System opens a drop-down menu.
		\item User navigates to \textit{Add} -$>$ \textit{Existing File}.
		\item System opens an \textit{Add Existing File} dialog window.
		\item User selects \textit{PC} or \textit{SQuire} or \textit{Github}.
		\item System updates the dialog to reflect the selected source.
		\item User navigates to the file's location and selects it.
		\item User clicks \textit{Add}.
		\item System adds the file to the project.
	\end{enumerate}
	\textbf{Alternatives:} 
	\begin{enumerate}
		\item Step 1: The user right clicks in the project pane and the system continues on to step 2 above.
		\item Step 5-7: The user clicks \textit{Cancel} and a new file is not added to the project.
	\end{enumerate}
	\textbf{Postconditions:}
	\begin{enumerate}
		\item An existing file is added to the project.
		\item The database is updated to reflect the changes.
	\end{enumerate}
	\textbf{Related:} Add New File to Project
\end{framed} 

\newpage

\subsection{Delete File}

\begin{framed}
	\noindent\textbf{Task Name:} Delete File \\ \\
	\textbf{Task Category:} Project Management \\ \\
	\textbf{Summary:} The user performs this task to delete a file from the project. \\ \\
	\textbf{Preconditions:} 
	\begin{enumerate}
		\item User must be registered.
		\item User must be logged in.
		\item User must have a project open.
		\item The project must have at least one file.
	\end{enumerate}
	\textbf{Steps:}
	\begin{enumerate}
		\item User clicks right clicks a file in the project pane.
		\item System opens a drop-down menu.
		\item User navigates to \textit{Delete}.
		\item System opens an \textit{Delete File} dialog window, asking if the user is sure.
		\item User selects \textit{Yes}.
		\item System deletes the file from the project.
	\end{enumerate}
	\textbf{Alternatives:} 
	\begin{enumerate}
		\item Step 5: The user clicks \textit{Cancel} instead and the file is not deleted from the project.
	\end{enumerate}
	\textbf{Postconditions:}
	\begin{enumerate}
		\item The file is deleted from the project.
		\item The database is updated to reflect the changes.
	\end{enumerate}
\end{framed} 

\newpage

\subsection{Export Project}

\begin{framed}
	\noindent\textbf{Task Name:} Export Project \\ \\
	\textbf{Task Category:} Project Management \\ \\
	\textbf{Summary:} The user performs this task to download a project as a zip file. \\ \\
	\textbf{Preconditions:} 
	\begin{enumerate}
		\item User must be registered.
		\item User must be logged in.
		\item User must have a project open.
		\item User must have download permissions.
	\end{enumerate}
	\textbf{Steps:}
	\begin{enumerate}
		\item User clicks \textit{File} in the top menu bar.
		\item System opens a drop-down menu.
		\item User navigates to \textit{Export} -$>$ \textit{Project}.
		\item System opens an \textit{Export} dialog window.
		\item User navigates to the export location.
		\item User clicks \textit{Export}.
		\item System zips the file and downloads it to the specified location.
	\end{enumerate}
	\textbf{Alternatives:} 
	\begin{enumerate}
		\item Step 1: The user right clicks in the project pane and the system continues on to step 2 above.
		\item Step 5: The user clicks \textit{Cancel} and the project is not exported.
	\end{enumerate}
	\textbf{Related:} Export Files
\end{framed}

\newpage

\subsection{Export Files}

\begin{framed}
	\noindent\textbf{Task Name:} Export Files \\ \\
	\textbf{Task Category:} Project Management \\ \\
	\textbf{Summary:} The user performs this task to download a number of files from a project. \\ \\
	\textbf{Preconditions:} 
	\begin{enumerate}
		\item User must be registered.
		\item User must be logged in.
		\item User must have a project open.
		\item Must have at least one file in the project.
		\item User must have download permissions.
	\end{enumerate}
	\textbf{Steps:}
	\begin{enumerate}
		\item User clicks \textit{File} in the top menu bar.
		\item System opens a drop-down menu.
		\item User navigates to \textit{Export} -$>$ \textit{Files}.
		\item System opens an \textit{Export} dialog window showing the project files on the left pane and the export location in the right pane.
		\item User selects a number of files on the left pane.
		\item User navigates to the export location in the right pane.
		\item User clicks \textit{Export}.
		\item System downloads the selected files to the specified location.
	\end{enumerate}
	\textbf{Alternatives:} 
	\begin{enumerate}
		\item Step 1: The user right clicks in the project pane and the system continues on to step 2 above.
		\item Step 5: User selects a folder and all files under that folder are selected.
		\item Step 5-6: The user clicks \textit{Cancel} and the project is not exported.
	\end{enumerate}
	\textbf{Related:} Export Project
\end{framed}

\newpage	
		
\section{Editor/IDE}

\subsection{Display Typing User}

\begin{framed}
	\noindent\textbf{Name:} Display Typing User \\ \\
	\textbf{Category:} Editor/IDE \\ \\
	\textbf{Summary:} As a user starts typing, the cursor changes color to that user’s specified color. It will also show a little box with their name in front of the cursor. \\ \\
	\textbf{Purpose:} Make it clear when and who is collaborating. \\ \\
	\textbf{Preconditions:} 
	\begin{enumerate}
		\item Must be registered.
		\item Must be logged in.
		\item A file is open.
		\item One or more users are connected to the same file.
	\end{enumerate}
	\textbf{Steps:}
	\begin{enumerate}
		\item User starts typing.
		\item System displays a separate cursor with the typing user’s name above it to other users.
	\end{enumerate}
\end{framed}

\newpage

\subsection{Display Author and Date}

\begin{framed}
	\noindent\textbf{Name:} Display Author and Date \\ \\
	\textbf{Category:} Editor/IDE \\ \\
	\textbf{Summary:} Display the authors and dates above each written line of code when this option is selected. \\ \\
	\textbf{Purpose:} Allow users to see who wrote what line of code and when. \\ \\
	\textbf{Preconditions:} 
	\begin{enumerate}
		\item User is registered.
		\item User is logged in.
	\end{enumerate}	
	\textbf{Steps:}
	\begin{enumerate}
		\item User clicks on the \textit{View} menu option in the menu bar.
		\item The system displays a drop-down menu with the option to Display Authors, among others.
		\item The user clicks on the Display Authors option.
		\item The editor will display the author and date in small font above each line of code.
	\end{enumerate}
	\textbf{Alternatives:} Same idea, but option to display user colors beside their code lines. When hovering over the color, it will display the username.
\end{framed}

\newpage

\subsection{Show Hightlighting}

\begin{framed}
	\noindent\textbf{Name:} Show Highlighting \\ \\
	\textbf{Category:} Editor/IDE \\ \\
	\textbf{Summary:} Shows users what other users are highlighting. \\ \\
	\textbf{Purpose:} Aid communication and collaboration by showing what other users are highlighting. \\ \\
	\textbf{Preconditions:} 
	\begin{enumerate}
		\item User has a file or project open.
		\item User is in a shared project.
		\item Another user is connected to the project.
		\item User has read permission.
	\end{enumerate}		
	\textbf{Steps:}
	\begin{enumerate}
		\item User highlights text in an open text file.
		\item System displays that user’s highlighting in a different color to other users in same file.
	\end{enumerate}	
\end{framed}
\newpage

\subsection{Open File in New Tab}

\begin{framed}
	\noindent\textbf{Name:} Open File in New Tab \\ \\
	\textbf{Category:} Editor/IDE \\ \\
	\textbf{Summary:} Allows users to open a file. \\ \\
	\textbf{Purpose:} Opening files is essential in being able to work on a project. \\ \\
	\textbf{Preconditions:} 
	\begin{enumerate}
		\item User is logged in.
		\item User has a project open.
		\item Project contains at least one file.
		\item User has read permission.
	\end{enumerate}		
	\textbf{Steps:}
	\begin{enumerate}
		\item User double clicks a file.
		\item The editor opens its contents in a new tab and focuses on it.
	\end{enumerate}	
	\textbf{Alternatives:} Step 1: Instead of double clicking a file, the user right clicks it and navigates to \textit{Open}.
\end{framed}
\newpage

\subsection{Close Tab}

\begin{framed}
	\noindent\textbf{Name:} Close Tab \\ \\
	\textbf{Category:} Editor/IDE \\ \\
	\textbf{Summary:} Allows users to close a tab. \\ \\
	\textbf{Purpose:} Closing tabs is essential for increasing efficiency and productivity. \\ \\
	\textbf{Preconditions:} 
	\begin{enumerate}
		\item User is logged in.
		\item User has a project open.
		\item A file is open.
		\item User has read permission.
	\end{enumerate}		
	\textbf{Steps:}
	\begin{enumerate}
		\item User clicks the `x' on a tab.
		\item The editor closes the tab.
	\end{enumerate}	
	\textbf{Alternatives:} Step 1: The user right-clicks the tab and selects \textit{Close}. \\ \\
	\textbf{Related:} Close All Tabs.
\end{framed}

\end{document}