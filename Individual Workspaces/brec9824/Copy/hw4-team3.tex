%%%% CS 383 HW #3 - hw3-team3.tex
%%%% Due on BBLearn before 10pm on Friday 2/19/2016


%##########################################################################################################################################################################
% Header
%##########################################################################################################################################################################

\documentclass[11pt]{report}

\usepackage{graphicx}
\usepackage{caption}

\marginparwidth 0.5in 
\oddsidemargin 0.25in 
\evensidemargin 0.25in 
\marginparsep 0.25in
\topmargin 0.0in 
\textwidth 6in \textheight 8.5in


\begin{document}



%##########################################################################################################################################################################
% Use Case Descriptions
%##########################################################################################################################################################################

\chapter{Use Case Descriptions}

\section{Authentication (mora5651)}
\subsection{Sign up (mora5651)}
\begin{tabular}{ p{2cm} p{12cm} }
 \hline
 \\
 \textit{Actors:} & User. \\ 
 \\
 \textit{Goals:} & To register and create an account in sQuire. \\
 \\
 \textit{Pre-conditions:} & None. \\
 \\
 \textit{Summary:} & The user signs up and creates an account using their email address and, creates a username and password. \\ 
 \\
 \textit{Related use cases:} & None. \\ 
 \\
 \textit{Steps:} & \begin{enumerate}
  \item User is prompted to enter email, username and password. 
  \item System sends confirmation email. 
  \item User verifies email. 
  \item System saves information, and redirectes user to sign in page. 
 \end{enumerate} \\
 \\
 \textit{Alternative 1:} & User already has an account. \\ 
 \\
 \textit{Alternative 2:} & User doesn't confirm email. Delete request after timeout period. \\
 \\
\hline
\end{tabular}

\subsection{Sign in (mora5651)}
\begin{tabular}{ p{2cm} p{12cm} }
 \hline
 \\
 \textit{Actors:} & Users. \\ 
 \\
 \textit{Goals:} & Pre-existing user signs into profile. \\
 \\
\textit{Precondition:} & User must already have an account \\
\\
\textit{Summary:} & User wishes to access their account, projects and info. \\
\\
 \textit{Steps:} & \begin{enumerate}
  \item User is prompted to enter username/e-mail, and password. 
  \item System verifies information
  \item Correct information prompts user to their home page. 
  \item If user forgot their user name, they can click "Forgot Username button. They then input their email address.
  \item System validates their email address with an account, and sends an email with the username  
  \item If user forgot their password, they click "Forgot password" button. Then they input their email address.
  \item System validates their email address with an account, and sends an email to them with a temporary password.
 \end{enumerate} \\
 \\
 \textit{Alternatives:} & Information is incorrect, user tries again. Or makes a new account \\
 \\
\hline
\end{tabular}

\subsection{Logout (mora5651)}
\begin{tabular}{ p{2cm} p{12cm} }
 \hline
 \\
 \textit{Actors:} & Users. \\ 
  \\
 \textit{Goal:} & Existing user logs out \\ 
 \\
 \textit{Precondition:} & User must be logged in \\
 \\
 \textit{Summary:}  & The user can log-out of the program at any time. \\
 \\
 \textit{Steps:} & \begin{enumerate}
 \item User clicks the "log-out" button. 
 \item System prompts user to ensure all unsaved work has been saved. 
 \item User verifies. 
 \item System logs user out. 
 \end{enumerate} \\
\\
  \textit{Alternative 1:} & The program will send notification to ask if the user is sure to sign out. \\
  \textit{Alternative 2:} & User cancels on step two. Return to home page. \\ 
 \\
\hline
\end{tabular}

\subsection{Forgotten Account Information (mora5651)}
\begin{tabular}{ p{2cm} p{12cm} }
 \hline
 \\
 \textit{Actors:} & Users. \\ 
  \\
 \textit{Goal:} & Recover forgotten account information. \\ 
 \\
 \textit{Precondition:} & User must already have an account. \\
 \\
 \textit{Summary:}  & User has forgotten their account information, and wishes to recover information. \\
 \\
 \textit{Steps:} & \begin{enumerate}
 \item User clicks the "Forgotten username/password" button. 
 \item User inputs their email address. 
 \item System validates their email address with an account, and sends
an email with the username or password reset. \\ 
 \item System logs user out. 
 \end{enumerate} \\
\\
  \textit{Alternative 1:} & Information is incorrect, user tries again. Or makes a new account. \\
\\
\hline
\end{tabular}


\section{Project Ideas (dani2918)}
\subsection{Browse Project Ideas (dani2918)}
\begin{tabular}{ p{2cm} p{12cm} }
 \hline
 \\
 \textit{Actors:} & User \\ 
 \\
 \textit{Goals:} & Examine a list of available project ideas  \\
 \\
 \textit{Pre-conditions:} & User is signed in  \\
 \\
 \textit{Summary:} & User looks through posted project ideas to find projects to work on/discuss \\ 
 \\
 \textit{Related use cases:} & Comment on Project Idea, Vote on Project Idea, Share Project Idea \\ 
 \\
 \textit{Steps:} & \begin{enumerate}
  \item Actor selects Browse Project Ideas button
  \item Actor refines search by selecting from list of project categories as desired
  \item Actor enters terms into search field as desired and views a list of top projects
  \item Actor selects desired project
  \item System displays detailed project information
 \end{enumerate} \\
 \\
 \textit{Alternatives:} & None. \\
 \\
 \textit{Post-conditions:} & None. \\
 \\
\hline
\end{tabular}

\subsection{Create Project Idea Thread (dani2918)}
\begin{tabular}{ p{2cm} p{12cm} }
 \hline
 \\
 \textit{Actors:} & Project Administrator \\ 
 \\
 \textit{Goals:} & Generate public interest in project idea  \\
 \\
 \textit{Pre-conditions:} & Prospective project administrator is signed in  \\
 \\
 \textit{Summary:} &  A user with interest in heading up own project can post ideas to get feedback and/or recruit collaborators \\ 
 \\
 \textit{Related use cases:} & Manage Project Idea Thread \\ 
 \\
 \textit{Steps:} & \begin{enumerate}
  \item Actor selects Create Project Idea button
  \item Actor enters prospective project title and thoughts and ideas as a description
  \item Actor selects Submit button

 \end{enumerate} \\
 \\
 \textit{Alternatives:} & None. \\
 \\
 \textit{Post-conditions:} & None. \\
 \\
\hline
\end{tabular}

\subsection{Manage Project Idea Thread (dani2918)}
\begin{tabular}{ p{2cm} p{12cm} }
 \hline
 \\
 \textit{Actors:} & Project Administrator \\ 
 \\
 \textit{Goals:} & Respond to feedback and manage project idea  \\
 \\
 \textit{Pre-conditions:} & Prospective project administrator is signed in and has navigated to one of his or her own project threads  \\
 \\
 \textit{Summary:} &  A prospective project administrator responds to others' questions/comments \\ 
 \\
 \textit{Related use cases:} & Create Project Idea Thread \\ 
 \\
 \textit{Steps:} & \begin{enumerate}
  \item Actor selects Reply button on a comment
  \item Actor types feedback to other user
  \item Actor selects Submit button 
  \item System shows confirmation that feedback was received 
  
 \end{enumerate} \\
 \\
 \textit{Alternatives:} & Actor selects Delete Comment instead of Reply to remove harmful feedback. \\
 \\
 \textit{Post-conditions:} & None. \\
 \\
\hline
\end{tabular}

\subsection{Comment on Project Idea (dani2918)}
\begin{tabular}{ p{2cm} p{12cm} }
 \hline
 \\
 \textit{Actors:} & User \\ 
 \\
 \textit{Goals:} & Provide detailed feedback on project ideas  \\
 \\
 \textit{Pre-conditions:} & Actor is signed in, has navigated to a project idea  \\
 \\
 \textit{Summary:} &  User provides feedback to or asks questions about a prospective project. \\ 
 \\
 \textit{Related use cases:} & Browse Project Ideas, Vote on Project Idea, Manage Project Idea Thread \\ 
 \\
 \textit{Steps:} & \begin{enumerate}
  \item Actor selects Comment button
  \item Actor types feedback into field 
  \item Actor clicks Submit button
  \item System shows confirmation that feedback was received 
 \end{enumerate} \\
 \\
 \textit{Alternatives:} & None \\
 \\
 \textit{Post-conditions:} & None. \\
 \\
\hline 
\end{tabular}

\subsection{Vote on Project Idea (dani2918)}
\begin{tabular}{ p{2cm} p{12cm} }
 \hline
 \\
 \textit{Actors:} & User \\ 
 \\
 \textit{Goals:} & Support promising project ideas or offer criticism to unfavorable ones  \\
 \\
 \textit{Pre-conditions:} & Actor is signed in, has navigated to a project idea  \\
 \\
 \textit{Summary:} &  User offers support/discourages a project idea so that prospective project administrators get feedback and promising project ideas get publicity \\ 
 \\
 \textit{Related use cases:} & Comment on Project Idea \\ 
 \\
 \textit{Steps:} & \begin{enumerate}
  \item Actor selects and Up Vote or Down Vote button
  \item Actor selects Submit button 
  \item System highlights which button the user has selected
 \end{enumerate} \\
 \\
 \textit{Alternatives:} & None \\
 \\
 \textit{Post-conditions:} & None. \\
 \\
\hline
\end{tabular}

\subsection{Share Project Idea (dani2918)}
\begin{tabular}{ p{2cm} p{12cm} }
 \hline
 \\
 \textit{Actors:} & User, Project Administrator \\ 
 \\
 \textit{Goals:} & Generate excitement about a project  \\
 \\
 \textit{Pre-conditions:} & Actor is signed in, has navigated to a project idea  \\
 \\
 \textit{Summary:} &  Prospective project administrators or users show other users which projects they believe are worthwhile \\ 
 \\
 \textit{Related use cases:} & Comment on Project Idea \\ 
 \\
 \textit{Steps:} & \begin{enumerate}
  \item Actor selects Share button
  \item Actor selects an audience and method with which to share selected project
  \item Actor selects Submit button
  \item System notifies or shows audience the shared project idea
 \end{enumerate} \\
 \\
 \textit{Alternatives:} & None \\
 \\
 \textit{Post-conditions:} & None. \\
 \\
\hline
\end{tabular}



\section{Communication (jank6275)}
\subsection{Open project chat (jank6275)}
\begin{tabular}{ p{2cm} p{12cm} }
 \hline
 \\
 \textit{Actors:} & User \\ 
 \\
 \textit{Goals:} & To open the project chat window. \\
 \\
 \textit{Pre-conditions:} & User must be registered, signed in, and have a open project.  \\
 \\
 \textit{Summary:} & User opens a project and the project chat automatically opens. The chat window displays chat history and updates when new chat messages are received. \\ 
 \\
 \textit{Related use cases:} & Join global chat. \\ 
 \\
 \textit{Steps:} & \begin{enumerate}
  \item User opens a project.
  \item Chat is notified that user has joined.
  \item System displays project chat window to the user.
 \end{enumerate} \\
 \\
 \textit{Alternatives:} & None. \\
 \\
 \textit{Post-conditions:} & None. \\
 \\
\hline
\end{tabular}

\subsection{Open global chat (jank6275)}
\begin{tabular}{ p{2cm} p{12cm} }
 \hline
 \\
 \textit{Actors:} & User \\ 
 \\
 \textit{Goals:} & To open the global chat window. \\
 \\
 \textit{Pre-conditions:} & User must be registered, signed in, and anywhere on website.  \\
 \\
 \textit{Summary:} & User authenticates with the server and the global chat automatically opens. The chat window displays chat history and updates when new chat messages are received.  \\ 
 \\
 \textit{Related use cases:} & Join project chat. \\ 
 \\
 \textit{Steps:} & \begin{enumerate}
  \item User clicks open global chat.
  \item Chat is notified that user has joined.
  \item System displays global chat window.
 \end{enumerate} \\
 \\
 \textit{Alternatives:} & None. \\
 \\
 \textit{Post-conditions:} & None. \\
 \\
\hline
\end{tabular}

\subsection{Close project chat (jank6275)}
\begin{tabular}{ p{2cm} p{12cm} }
 \hline
 \\
 \textit{Actors:} & User \\ 
 \\
 \textit{Goals:} & To close the project chat window. \\
 \\
 \textit{Pre-conditions:} & User must be registered, signed in, and in editor Mode.  \\
 \\
 \textit{Summary:} & User clicks on close project chat and the chat window closes. \\ 
 \\
 \textit{Related use cases:} & Close global chat. \\ 
 \\
 \textit{Steps:} & \begin{enumerate}
  \item User clicks close project chat.
  \item Chat is notified that user has left.
  \item Client closes project chat window.
 \end{enumerate} \\
 \\
 \textit{Alternatives:} & None. \\
 \\
 \textit{Post-conditions:} & None. \\
 \\
\hline
\end{tabular}

\subsection{Close global chat (jank6275)}
\begin{tabular}{ p{2cm} p{12cm} }
 \hline
 \\
 \textit{Actors:} & User \\ 
 \\
 \textit{Goals:} & To close the global chat window. \\
 \\
 \textit{Pre-conditions:} & User must be registered, signed in, and anywhere on website.  \\
 \\
 \textit{Summary:} & User clicks on open global chat and the chat opens, displaying chat history and updating when needed. \\ 
 \\
 \textit{Related use cases:} & Close project chat. \\ 
 \\
 \textit{Steps:} & \begin{enumerate}
  \item User clicks close global chat.
  \item Chat is notified that user has left.
  \item Client closes global chat window.
 \end{enumerate} \\
 \\
 \textit{Alternatives:} & None. \\
 \\
 \textit{Post-conditions:} & None. \\
 \\
\hline
\end{tabular}

\subsection{Write to project chat (jank6275)}
\begin{tabular}{ p{2cm} p{12cm} }
 \hline
 \\
 \textit{Actors:} & User \\ 
 \\
 \textit{Goals:} & To send text to project chat. \\
 \\
 \textit{Pre-conditions:} & User must be registered, signed in, a project opened, with the project chat window open, and the text box selected.  \\
 \\
 \textit{Summary:} & User clicks in the project chat text box and then types a message then either presses enter or clicks the submit button. The text is displayed to all users in the chat, including the user. \\ 
 \\
 \textit{Related use cases:} & Write to global chat. \\ 
 \\
 \textit{Steps:} & \begin{enumerate}
  \item User clicks in the project chat box.
  \item User types a message and then presses enter or clicks submit button.
  \item Message is relayed to all clients with project chat open.
  \item Message is displayed.
 \end{enumerate} \\
 \\
 \textit{Alternatives:} & None. \\
 \\
 \textit{Post-conditions:} & None. \\
 \\
\hline
\end{tabular}

\subsection{Write to global chat (jank6275)}
\begin{tabular}{ p{2cm} p{12cm} }
 \hline
 \\
 \textit{Actors:} & User \\ 
 \\
 \textit{Goals:} & To send text to global chat. \\
 \\
 \textit{Pre-conditions:} & User must be registered, signed in, anywhere on website, with the global chat window open, and the text box selected.  \\
 \\
 \textit{Summary:} & User clicks in the global chat text box and then types a message then either presses enter or clicks the submit button. The text is displayed to all users in the chat, including the user. \\ 
 \\
 \textit{Related use cases:} & Write to project chat. \\ 
 \\
 \textit{Steps:} & \begin{enumerate}
  \item User clicks in the global chat box.
  \item User types a message and then presses enter or clicks submit button.
  \item Message is relayed to all clients with global chat open.
  \item Message is displayed.
 \end{enumerate} \\
 \\
 \textit{Alternatives:} & None. \\
 \\
 \textit{Post-conditions:} & None. \\
 \\
\hline
\end{tabular}

\subsection{Modify chat font (jank6275)}
\begin{tabular}{ p{2cm} p{12cm} }
 \hline
 \\
 \textit{Actors:} & User \\ 
 \\
 \textit{Goals:} & To change a users font style inside the global and project chat. \\
 \\
 \textit{Pre-conditions:} & User must be registered, signed in, the user settings window opened, and the chat settings tab open.  \\
 \\
 \textit{Summary:} & The user clicks the settings menu and changes their font style for both the project and global chat through a drop down box of available fonts. \\ 
 \\
 \textit{Related use cases:} & Modify chat color. \\ 
 \\
 \textit{Steps:} & \begin{enumerate}
  \item User clicks the settings menu.
  \item User clicks chat settings tab.
  \item User clicks chat font drop down box.
  \item User clicks desired font.
  \item User clicks save.
  \item The user\'s selection is saved in the database.
  \item All further chat messages will use the selected font.
 \end{enumerate} \\
 \\
 \textit{Alternatives:} & None. \\
 \\
 \textit{Post-conditions:} & None. \\
 \\
\hline
\end{tabular}

\subsection{Modify chat color (jank6275)}
\begin{tabular}{ p{2cm} p{12cm} }
 \hline
 \\
 \textit{Actors:} & User \\ 
 \\
 \textit{Goals:} & To change a users font color inside the global and project chat. \\
 \\
 \textit{Pre-conditions:} & User must be registered, signed in, the user settings window opened, and the chat settings tab open.  \\
 \\
 \textit{Summary:} & The user clicks the settings menu and changes their font color for both the project and global chat through a drop down box of available colors. \\ 
 \\
 \textit{Related use cases:} & Modify chat font. \\ 
 \\
 \textit{Steps:} & \begin{enumerate}
  \item User clicks the settings menu.
  \item User clicks chat settings tab.
  \item User clicks chat color drop down box.
  \item User clicks desired color.
  \item User clicks save.
  \item The user\'s selection is saved in the database.
  \item All further chat messages from the user will use the selected color.
 \end{enumerate} \\
 \\
 \textit{Alternatives:} & None. \\
 \\
 \textit{Post-conditions:} & None. \\
 \\
\hline
\end{tabular}



\newpage

\subsection{File Management (snev7821)}
\subsection{Create New Project (snev7821)}
\begin{framed}

	\textbf{Task Category:} File Management \\ \\
	\textbf{Actor:} User \\ \\
	\textbf{Summary:} The user performs this task to create a new project. \\ \\
	\textbf{Preconditions:} 
	\begin{enumerate}
		\item User must be registered.
		\item User must be logged in.
	\end{enumerate}
	\textbf{Steps:}
	\begin{enumerate}
		\item User clicks \textit{Projects} in the top menu bar.
		\item System opens a drop-down menu.
		\item User navigates to \textit{Add} -$>$ \textit{New Project}.
		\item System opens an \textit{Add New Project} dialog window.
		\item User provides a title and optionally description.
		\item User clicks \textit{Add}.
		\item System adds the file(s) to the project.
	\end{enumerate}
	\textbf{Alternatives:} 
	\begin{enumerate}
		\item Step 1: The user right clicks in the project panel and the system continues on to step 2 above.
		\item Step 5: The user clicks \textit{Cancel} and a new project is not added.
	\end{enumerate}
	\textbf{Postconditions:}
	\begin{enumerate}
		\item A new project is added to the project.
		\item The database is updated to reflect the changes.
	\end{enumerate}
	\textbf{Related:} Add Existing File to Project, Add New File to Project
\end{framed} 

\newpage




\subsubsection{Add New File to Project (wern0096) modified for hw4 (snev7821)}
\begin{framed}

	\textbf{Task Category:} File Management \\ \\
	\textbf{Actor:} User \\ \\
	\textbf{Summary:} The user performs this task to add a new file to the project. \\ \\
	\textbf{Preconditions:} 
	\begin{enumerate}
		\item User must be registered.
		\item User must be logged in.
		\item User must have a project open.
	\end{enumerate}
	\textbf{Steps:}
	\begin{enumerate}
		\item User clicks \textit{File} in the top menu bar.
		\item System opens a drop-down menu.
		\item User navigates to \textit{Add} -$>$ \textit{New File}.
		\item System opens an \textit{Add New File} dialog window.
		\item User selects the file type and names the file.
		\item User clicks \textit{Add}.
		\item System adds the file to the project.
	\end{enumerate}
	\textbf{Alternatives:} 
	\begin{enumerate}
		\item Step 1: The user right clicks in the project panel and the system continues on to step 2 above.
		\item Step 5: The user clicks \textit{Cancel} and a new file is not added to the project.
	\end{enumerate}
	\textbf{Postconditions:}
	\begin{enumerate}
		\item A new file is added to the project.
		\item The database is updated to reflect the changes.
	\end{enumerate}
	\textbf{Related:} Add Existing File to Project
\end{framed} 

\newpage

\subsubsection{Add Existing File to Project (wern0096) modified for hw4 (snev7821)}
\begin{framed}

	\textbf{Task Category:} File Management \\ \\
	\textbf{Actor:} User \\ \\
	\textbf{Summary:} The user performs this task to add an existing file to the project. \\ \\
	\textbf{Preconditions:} 
	\begin{enumerate}
		\item User must be registered.
		\item User must be logged in.
		\item User must have a project open.
	\end{enumerate}
	\textbf{Steps:}
	\begin{enumerate}
		\item User clicks \textit{File} in the top menu bar.
		\item System opens a drop-down menu.
		\item User navigates to \textit{Add} -$>$ \textit{Existing File}.
		\item System opens an \textit{Add Existing File} dialog window.
		\item User selects \textit{PC} or \textit{SQuire} or \textit{Github}.
		\item System updates the dialog to reflect the selected source.
		\item User navigates to the file's location and selects it.
		\item User clicks \textit{Add}.
		\item System adds the file to the project.
	\end{enumerate}
	\textbf{Alternatives:} 
	\begin{enumerate}
		\item Step 1: The user right clicks in the project panel and the system continues on to step 2 above.
		\item Step 5-7: The user clicks \textit{Cancel} and a new file is not added to the project.
	\end{enumerate}
	\textbf{Postconditions:}
	\begin{enumerate}
		\item An existing file is added to the project.
		\item The database is updated to reflect the changes.
	\end{enumerate}
	\textbf{Related:} Add New File to Project
\end{framed} 

\newpage

\subsubsection{Delete File (wern0096) modified for hw4 (snev7821)}
\begin{framed}

	\textbf{Task Category:} File Management \\ \\
	\textbf{Actor:} User \\ \\
	\textbf{Summary:} The user performs this task to delete a file from the project. \\ \\
	\textbf{Preconditions:} 
	\begin{enumerate}
		\item User must be registered.
		\item User must be logged in.
		\item User must have a project open.
		\item User must be administrator of project.
		\item Current project must have at least one file.
	\end{enumerate}
	\textbf{Steps:}
	\begin{enumerate}
		\item User right clicks a file in the project pane.
		\item System opens a drop-down menu.
		\item User navigates to \textit{Delete}.
		\item System opens an \textit{Delete File} dialog window, asking if the user is sure.
		\item User selects \textit{Yes}.
		\item System deletes the file from the project.
	\end{enumerate}
	\textbf{Alternatives:} 
	\begin{enumerate}
		\item Step 5: The user clicks \textit{Cancel} instead and the file is not deleted from the project.
		\item The user selects multiple files before step 1.
	\end{enumerate}
	\textbf{Postconditions:}
	\begin{enumerate}
		\item The file is deleted from the project.
		\item The database is updated to reflect the changes.
	\end{enumerate}
		\textbf{Related:} Delete Project
\end{framed} 
\newpage


\subsubsection{Delete Project (snev7821)}
\begin{framed}

	\textbf{Task Category:} File Management \\ \\
	\textbf{Actor:} User \\ \\
	\textbf{Summary:} The user performs this task to delete a project. \\ \\
	\textbf{Preconditions:} 
	\begin{enumerate}
		\item User must be registered.
		\item User must be logged in.
		\item User must have a project open.
		\item User must be sole administrator of project.
	\end{enumerate}
	\textbf{Steps:}
	\begin{enumerate}
		\item User clicks \textit{Projects} tab.
		\item System opens a drop-down menu.
		\item User navigates to \textit{Delete}.
		\item System opens an \textit{Delete Project} dialog window, asking if the user is sure.
		\item User selects \textit{Yes}.
		\item System deletes the project and related files from system.
	\end{enumerate}
	\textbf{Alternatives:} 
	\begin{enumerate}
		\item Step 5: The user clicks \textit{Cancel} instead and the project is not deleted.
	\end{enumerate}
	\textbf{Postconditions:}
	\begin{enumerate}
		\item The project (and related files) are deleted.
		\item The database is updated to reflect the changes.
	\end{enumerate}
		\textbf{Related:} Delete File
\end{framed} 
\newpage

\subsubsection{Export Project (wern0096) modified for hw4 (snev7821)}
\begin{framed}

	\textbf{Task Category:} File Management \\ \\
	\textbf{Actor:} User \\ \\
	\textbf{Summary:} The user performs this task to download a project as a zip file. \\ \\
	\textbf{Preconditions:} 
	\begin{enumerate}
		\item User must be registered.
		\item User must be logged in.
		\item User must have a project open.
		\item User must have download permissions.
	\end{enumerate}
	\textbf{Steps:}
	\begin{enumerate}
		\item User clicks \textit{File} in the top menu bar.
		\item System opens a drop-down menu.
		\item User navigates to \textit{Export} -$>$ \textit{Project}.
		\item System opens an \textit{Export} dialog window.
		\item User navigates to the export location.
		\item User clicks \textit{Export}.
		\item System zips the file and downloads it to the specified location.
	\end{enumerate}
	\textbf{Alternatives:} 
	\begin{enumerate}
		\item Step 1: The user right clicks in the project panel and the system continues on to step 2 above.
		\item Step 5: The user clicks \textit{Cancel} and the project is not exported.
	\end{enumerate}
	\textbf{Related:} Export Files
\end{framed}

\newpage

\subsubsection{Export Files (wern0096) modified for hw4 (snev7821)}
\begin{framed}

	\textbf{Task Category:} File Management \\ \\
	\textbf{Actor:} User \\ \\
	\textbf{Summary:} The user performs this task to download a number of files from a project. \\ \\
	\textbf{Preconditions:} 
	\begin{enumerate}
		\item User must be registered.
		\item User must be logged in.
		\item User must have a project open.
		\item Must have at least one file in the project.
		\item User must have download permissions.
	\end{enumerate}
	\textbf{Steps:}
	\begin{enumerate}
		\item User clicks \textit{File} in the top menu bar.
		\item System opens a drop-down menu.
		\item User navigates to \textit{Export} -$>$ \textit{Files}.
		\item System opens an \textit{Export} dialog window showing the project files on the left panel and the export location in the right panel.
		\item User selects a number of files on the left pane.
		\item User navigates to the export location in the right pane.
		\item User clicks \textit{Export}.
		\item System downloads the selected files to the specified location.
	\end{enumerate}
	\textbf{Alternatives:} 
	\begin{enumerate}
		\item Step 1: The user right clicks in the project panel and the system continues on to step 2 above.
		\item Step 5: User selects a folder and all files under that folder are selected.
		\item Step 5-6: The user clicks \textit{Cancel} and the project is not exported.
	\end{enumerate}
	\textbf{Related:} Export Project
\end{framed}

\newpage

\subsubsection{Open File in New Tab (wern0096) modified for hw4 (snev7821)}
\begin{framed}

	\textbf{Category:} File Management \\ \\
	\textbf{Actor:} User \\ \\
	\textbf{Summary:} Allows users to open a file. \\ \\
	\textbf{Purpose:} Opening files is essential in being able to work on a project. \\ \\
	\textbf{Preconditions:} 
	\begin{enumerate}
		\item User is registered.
		\item User is logged in.
		\item User has a project open.
		\item Current project contains at least one file.
		\item User has read permission.
	\end{enumerate}		
	\textbf{Steps:}
	\begin{enumerate}
		\item User double clicks a file.
		\item The editor opens its contents in a new tab and focuses on it.
	\end{enumerate}	
	\textbf{Alternatives:} Step 1: Instead of double clicking a file, the user right clicks it and navigates to \textit{Open}.
\end{framed}
\newpage




\section{File Editing (snev7821)}
\subsection{View Line Numbers (wern0096) updated by (snev7821)}
\begin{framed}

	\textbf{Category:} File Editing \\ \\
	\textbf{Actor:} User \\ \\
	\textbf{Summary:} Allows the user to hide line numbers to the left of the document. \\ \\
	\textbf{Purpose:} In case user wants to hide line numbers so they have more space for text.\\ \\
	\textbf{Preconditions:} 
	\begin{enumerate}
		\item Must be registered.
		\item Must be logged in.
		\item User has view permission.
		\item A file is open.
		\item Line numbers are on
	\end{enumerate}
	\textbf{Steps:}
	\begin{enumerate}
		\item User selects the \textit{View} menu option.
		\item System displays a drop-down with various options.
		\item User selects the \textit{Hide Line Numbers} option.
		\item System hides line numbers to the left of the document.
	\end{enumerate}
	\textbf{Related:}
	\begin{enumerate}
		\item View References
		\item View Dates
		\item View Authors
	\end{enumerate}
\end{framed}

\newpage

\subsection{View References (wern0096) updated for hw4 by (snev7821)}
\begin{framed}

	\textbf{Category:} File Editing \\ \\
	\textbf{Actor:} User \\ \\
	\textbf{Summary:} Allows the user to view the number of references to a given function. \\ \\
	\textbf{Purpose:} It is useful to know the number of references to a given function for optimization and debugging purposes. \\ \\
	\textbf{Preconditions:} 
	\begin{enumerate}
		\item Must be registered.
		\item Must be logged in.
		\item User has view permission.
		\item A \textbf{code} file is open.
	\end{enumerate}
	\textbf{Steps:}
	\begin{enumerate}
		\item User selects the \textit{View} menu option.
		\item System displays a drop-down with various options.
		\item User selects the \textit{View References} option.
		\item System displays the number of references above each method declaration.
	\end{enumerate}
	\textbf{Related:}
	\begin{enumerate}
		\item Hide Line Numbers
		\item View Dates
		\item View Authors
	\end{enumerate}
\end{framed}

\newpage

\subsection{View Dates (wern0096) updated by (snev7821)}
\begin{framed}

	\textbf{Category:} File Editing \\ \\
	\textbf{Actor:} User \\ \\
	\textbf{Summary:} Allows the user to view the last date that each block of a document was edited. Blocks are defined as any number of lines that was written by a single user. Minimum block size is one line. \\ \\
	\textbf{Purpose:} This provides a useful metric for how up-to-date parts of the document are. \\ \\
	\textbf{Preconditions:} 
	\begin{enumerate}
		\item Must be registered.
		\item Must be logged in.
		\item User has view permission.
		\item A file is open.
	\end{enumerate}
	\textbf{Steps:}
	\begin{enumerate}
		\item User selects the \textit{View} menu option.
		\item System displays a drop-down with various options.
		\item User selects the \textit{View Dates} option.
		\item System displays the last date that each block of a document was edited.
	\end{enumerate}
	\textbf{Related:}
	\begin{enumerate}
		\item Hide Line Numbers
		\item View References
		\item View Authors
	\end{enumerate}
\end{framed}

\newpage

\subsection{View Authors (wern0096) updated by (snev7821)}
\begin{framed}

	\textbf{Category:} File Editing \\ \\
	\textbf{Actor:} User \\ \\
	\textbf{Summary:} Allows the user to view the last author that edited each block of a document. Blocks are defined as any number of lines that was written by a single user. Minimum block size is one line. \\ \\
	\textbf{Purpose:} This is an accountability tool allowing other users to identify who is responsible for a change to a document. \\ \\
	\textbf{Preconditions:} 
	\begin{enumerate}
		\item Must be registered.
		\item Must be logged in.
		\item User has read permission.
		\item A file is open.
	\end{enumerate}
	\textbf{Steps:}
	\begin{enumerate}
		\item User selects the \textit{View} menu option.
		\item System displays a drop-down with various options.
		\item User selects the \textit{View Authors} option.
		\item System displays the name of the last editor of each line of the document.
	\end{enumerate}
	\textbf{Related:}
	\begin{enumerate}
		\item Hide Line Numbers
		\item View References
		\item View Dates
	\end{enumerate}
\end{framed}




%2-24-16

\newpage

\subsection{Format Document (wern0096) updated by (snev7821)}
\begin{framed}

	\textbf{Category:} File Editing \\ \\
	\textbf{Actor:} User \\ \\
	\textbf{Summary:} Allows the user to format the document to a specified format \\ \\
	\textbf{Purpose:} An easy tool for making sweeping changes to a large part of a document. \\ \\
	\textbf{Preconditions:} 
	\begin{enumerate}
		\item Must be registered.
		\item Must be logged in.
		\item User has read/write permission.
		\item A file is open.
		\item The document has formatting options set.
	\end{enumerate}
	\textbf{Steps:}
	\begin{enumerate}
		\item User selects the \textit{Edit} menu option.
		\item System displays a drop-down with various options.
		\item User selects the \textit{Format Document} option.
		\item System formats the current document to the formatting settings currently set.
	\end{enumerate}
	\textbf{Alternatives:}
	\begin{enumerate}
		\item If no formatting settings are currently set, display a dialog box after step 3 and give the option for the user to do so now.
	\end{enumerate}
	\textbf{Related:}
	\begin{enumerate}
		\item Find/Replace
		\item Comment Section
	\end{enumerate}
\end{framed}

\newpage

\subsection{Find/Replace (wern0096) updated by (snev7821)}
\begin{framed}

	\textbf{Category:} File Editing \\ \\
	\textbf{Actor:} User \\ \\
	\textbf{Summary:} Allows the user to find and/or replace phrases. \\ \\
	\textbf{Purpose:} This is a powerful tool that allows a user to make safer, quicker, and more efficient changes to a document. \\ \\
	\textbf{Preconditions:} 
	\begin{enumerate}
		\item Must be registered.
		\item Must be logged in.
		\item User has read/write permission.
		\item A file is open.
	\end{enumerate}
	\textbf{Steps:}
	\begin{enumerate}
		\item User selects the \textit{Edit} menu option.
		\item System displays a drop-down with various options.
		\item User selects the \textit{Find/Replace} option.
		\item System displays a small form in an unobtrusive location.
		\item User enter the phrase to find and selects find.
		\item System highlights and focuses on the first occurrence of the phrase and all highlights all other occurrences.
	\end{enumerate}
	\textbf{Alternatives:}
	\begin{enumerate}
		\item User selects option to replace in step 5 and enters a phrase with which to replace the found occurrences of the searched phrase. The system replaces each occurrence.
	\end{enumerate}
	\textbf{Related:}
	\begin{enumerate}
		\item Format Document
		\item Find/Replace
	\end{enumerate}
\end{framed}

\newpage

\subsection{Comment Section (wern0096) updated by (snev7821)}
\begin{framed}

	\textbf{Category:} File Editing \\ \\
	\textbf{Actor:} User \\ \\
	\textbf{Summary:} Allows the user to comment out a part of a document. \\ \\
	\textbf{Purpose:} A useful and quick way to disable a large part of a document. \\ \\
	\textbf{Preconditions:} 
	\begin{enumerate}
		\item Must be registered.
		\item Must be logged in.
		\item A file is open.
		\item User has read/write permission.
		\item Current open document supports commenting.
	\end{enumerate}
	\textbf{Steps:}
	\begin{enumerate}
		\item User selects the \textit{Edit} menu option.
		\item System displays a drop-down with various options.
		\item User selects the \textit{Comment Section} option.
		\item System comments the selected area.
	\end{enumerate}
	\textbf{Alternatives:}
	\begin{enumerate}
		\item If document does not support commenting, display a dialog box telling the user.
	\end{enumerate}
	\textbf{Related:}
	\begin{enumerate}
		\item Format Document
		\item Find/Replace
	\end{enumerate}
\end{framed}

\newpage

\subsection{Display Typing User (wern0096) updated by (snev7821)}
\begin{framed}

	\textbf{Category:} File Editing \\ \\
	\textbf{Actors:} 
	\begin{enumerate}
		\item User
		\item Other Users
	\end{enumerate}
	\textbf{Summary:} As the user types, the system displays their name, their typing, and their caret, in a different color, to other users. \\ \\
	\textbf{Purpose:} Differentiate who is typing what. \\ \\
	\textbf{Preconditions:} 
	\begin{enumerate}
		\item Must be registered.
		\item Must be logged in.
		\item User has read/write permission.
		\item A file is open.
		\item Other users have the same document open.
	\end{enumerate}
	\textbf{Steps:}
	\begin{enumerate}
		\item User begins typing.
		\item System displays the user's typing, the user's name, and the user's caret, in a different color, to Other Users.
		\item Other Users see User typing, his username, and his caret, in a different color.
	\end{enumerate}
\end{framed}

\newpage

\subsection{Display Syntax Highlighting (wern0096) updated by (snev7821)}
\begin{framed}

	\textbf{Category:} File Editing \\ \\
	\textbf{Actor:} User \\ \\
	\textbf{Summary:} As the user types code, the editor will change font color for different code structures and keywords. \\ \\
	\textbf{Purpose:} Aids the user is writing code and identifying key code parts. \\ \\
	\textbf{Preconditions:} 
	\begin{enumerate}
		\item Must be registered.
		\item Must be logged in.
		\item User has read/write permission.
		\item A supported code file is open.
	\end{enumerate}
	\textbf{Steps:}
	\begin{enumerate}
		\item User begins typing.
		\item System automatically colors special code structures and keywords.
	\end{enumerate}
	\textbf{Related:} Display Syntax Errors
\end{framed}

\newpage

\section{User Prefrences (snev7821)}
\subsection{View User Preferences (snev7821)}
\begin{framed}
	\noindent\textbf{Name:} View User Preferences \\ \\
	\textbf{Category:} User Preferences - Editor  \\ \\
	\textbf{Actor:} User \\ \\
	\textbf{Summary:} User views their preferences and from here can change them \\ \\
	\textbf{Purpose:} Allows user to view their preferences and change them \\ \\
	\textbf{Preconditions:} 
	\begin{enumerate}
		\item Must be registered
		\item Must be logged in
		\item User is on user homepage
	\end{enumerate}
	\textbf{Steps:}
	\begin{enumerate}
		\item User clicks "Manage Editor Preferences"
		\item System presents user with preferences page
	\end{enumerate}
	\textbf{Related:} Modify chat font, Modify chat color, Edit user color
\end{framed}

\newpage

\subsection{Modify Chat Font (snev7821)}
\begin{framed}
	\noindent\textbf{Name:} Modify Chat Font \\ \\
	\textbf{Category:} User Preferences - Editor  \\ \\
	\textbf{Actor:} User \\ \\
	\textbf{Summary:} User changes the chat font \\ \\
	\textbf{Purpose:} Allows user to change what chat font they see for themselves and others \\ \\
	\textbf{Preconditions:} 
	\begin{enumerate}
		\item Must be registered
		\item Must be logged in
		\item User is on user editor preferences page
	\end{enumerate}
	\textbf{Steps:}
	\begin{enumerate}
		\item User clicks "Modify Chat Fonts" button
		\item System brings up list of fonts, for the user and others
		\item User selects a font for self
		\item User sets a font for others
		\item System saves user choices after each user action
	\end{enumerate}
	\textbf{Related:} Modify chat color, Edit user editor theme
\end{framed}

\newpage

\subsection{Modify Chat Color (snev7821)}
\begin{framed}
	\noindent\textbf{Name:} Modify Chat Color \\ \\
	\textbf{Category:} User Preferences - Editor  \\ \\
	\textbf{Actor:} User \\ \\
	\textbf{Summary:} User changes the chat color \\ \\
	\textbf{Purpose:} Allows user to change what chat color they see for themselves and others \\ \\
	\textbf{Preconditions:} 
	\begin{enumerate}
		\item Must be registered
		\item Must be logged in
		\item User is on user editor preferences page
	\end{enumerate}
	\textbf{Steps:}
	\begin{enumerate}
		\item User clicks "Modify Chat Colors" button
		\item System brings up list of colors, for the user and others
		\item User selects a color for self
		\item User sets a color for others
		\item System saves user choices after each user action
	\end{enumerate}
	\textbf{Related:} Modify chat font, Edit user editor theme
\end{framed}

\newpage

\subsection{Edit User Editor Theme (snev7821)}
\begin{framed}
	\noindent\textbf{Name:} Edit User Editor Theme \\ \\
	\textbf{Category:} User Preferences - Editor  \\ \\
	\textbf{Actor:} User \\ \\
	\textbf{Summary:}User changes the Editor theme \\ \\
	\textbf{Purpose:} Allows user to change theme of the collaborative editor \\ \\
	\textbf{Preconditions:} 
	\begin{enumerate}
		\item Must be registered
		\item Must be logged in
		\item User is on user editor preferences page
	\end{enumerate}
	\textbf{Steps:}
	\begin{enumerate}
		\item User clicks "Modify Editor Theme" button
		\item System brings up list of themes for editor
		\item User selects a theme
		\item System saves user selection
	\end{enumerate}
	\textbf{Related:} Modify chat font, Edit user color 
\end{framed}

\newpage

\subsection{Turn Off Global Chat (snev7821)}
\begin{framed}
	\noindent\textbf{Name:} Turn Off Global Chat \\ \\
	\textbf{Category:} User Preferences - Editor  \\ \\
	\textbf{Actor:} User \\ \\
	\textbf{Summary:} User turns off global chat \\ \\
	\textbf{Purpose:} Allows user to choose whether or not to engage in global chat \\ \\
	\textbf{Preconditions:} 
	\begin{enumerate}
		\item Must be registered
		\item Must be logged in
		\item User is on user editor preferences page
	\end{enumerate}
	\textbf{Steps:}
	\begin{enumerate}
		\item User checks "turn global chat off" box
		\item System brings up warning, explaining what this does
		\item User selects accept
		\item System saves user selection
		\item System disconnects user from global chat
	\end{enumerate}
	\textbf{Related:} Global chat
\end{framed}

\newpage

\subsection{Psuedo-offline Mode (snev7821)}
\begin{framed}
	\noindent\textbf{Name:} Psuedo-offline Mode \\ \\
	\textbf{Category:} User Preferences - Editor  \\ \\
	\textbf{Actor:} User \\ \\
	\textbf{Summary:} User changes to offline mode \\ \\
	\textbf{Purpose:} Allows user to turn off online features, including chat, public profiles, etc. Site then serves as basic editing environment\\ \\
	\textbf{Preconditions:} 
	\begin{enumerate}
		\item Must be registered
		\item Must be logged in
		\item User is on user editor preferences page
	\end{enumerate}
	\textbf{Steps:}
	\begin{enumerate}
		\item User clicks "offline mode" button
		\item System brings up warning
		\item User selects accept
		\item System saves user selection
		\item System closes chat
		\item System loads offline user page
		\item Upon disconnect with site, online mode restarts upon next connection
	\end{enumerate}
	\textbf{Related:} None 
\end{framed}

\section{Project Management}
\subsection{Create Project (Created by bolt1003)}
\begin{tabular}{ p{2cm} p{12cm} }
 \hline
 \\
 \textit{Actors:} & Users of sQuire. \\ 
 \\
 \textit{Goals:} & Create a Project. \\
 \\
 \textit{Pre-conditions:} & The user is logged in and at the bashboard. \\
 \\
 \textit{Summary:} & The user creates a project. \\ 
 \\
 \textit{Related use cases:} & None. \\ 
 \\
 \textit{Steps:} & \begin{enumerate}
  \item User selects the "+" icon and a wizard appears.
  \item A name is choosen for the project.
  \item Language is selected from a drop down menu.
  \item User clicks finish.
 \end{enumerate} \\
 \\
 \textit{Alternatives:} & Create project from the editor. \\
 \\
 \textit{Post-conditions:} & The user assigns permissions to access the project. \\
 \\
\hline
\end{tabular}

\subsection{Open a project (Created by bolt1003)}
\begin{tabular}{ p{2cm} p{12cm} }
 \hline
 \\
 \textit{Actors:} & Users of sQuire. \\ 
 \\
 \textit{Goals:} & Choose the desired project and open it. \\
 \\
 \textit{Pre-conditions:} & One or more projects are available, the user is logged in and at the dashboard. \\
 \\
 \textit{Summary:} & User looks through a list of projects and selects the desired project. \\ 
 \\
 \textit{Related use cases:} & None. \\ 
 \\
 \textit{Steps:} & \begin{enumerate}
  \item User clicks on projects in the menu bar.
  \item A list of projects appears and the user clicks on the desired project.
 \end{enumerate} \\
 \\
 \textit{Alternatives:} & Open a project from recent projects. \\
 \\
 \textit{Post-conditions:} & User closes sQuire. \\
 \\
\hline
\end{tabular}

\subsection{Join Project (Created by sass8427, Revised by bolt1003)}
\begin{tabular}{ p{2cm} p{12cm} }
\hline
\\
\textit{Actors:} & Users of sQuire. \\ 
\\
\textit{Goals:} & Join an existing project.\\
\\
\textit{Pre-conditions:} & Must be registered, logged in and have permission to join a project.\\
\\
\textit{Summary:} & The user logs in, chooses a project, and joins the project. \\
\\
\textit{Related use cases:} & Invite user to project, Accept user invite. \\
\\
\textit{Steps:} & \begin{enumerate}
 \item The user selects a project.
 \item The user chooses the "Join". 
 \item The project is added to the users projects bar.
 \item The user selects the project and selects "open".
\end{enumerate}\\
\\
\textit{Alternatives:} & User may decline an inventation to join a project. \\
\\
\textit{Post-conditions:} & None \\
\\
\hline
\end{tabular}

\subsection{Leave project (Created by sass8427, Revised by bolt1003)}
\begin{tabular}{ p{2cm} p{12cm} }
 \hline
 \\
 \textit{Actors:} & User \\ 
 \\
 \textit{Goals:} & Remove member status from project. \\
 \\
 \textit{Pre-conditions:} & Logged in, member of the respective project, not project owner.  \\
\\
 \textit{Summary:} & A member of a project can unjoin that project at any time as long as they are not the project owner. To prevent mistakenly unjoining a project, the user is asked to confirm their decision.\\ 
 \\
 \textit{Related use cases:} & \\ 
 \\
 \textit{Steps:} & \begin{enumerate}
  \item User selects a project.
  \item User clicks "Unjoin". 
  \item User is promted to confirm their decision
  \item User clicks "Confirm".
  \item User is removed from project member list.    
 \end{enumerate} \\
 \\
 \textit{Alternatives:} & User clicks "Cancel" at step 4, in which case the task is ends at that point. \\
 \\
 \textit{Post-conditions:} & None. \\
 \\
\hline
\end{tabular}

\subsection{Delete Project(Created by mora5651, Revised by bolt1003)}
\begin{tabular}{ p{2cm} p{12cm} }
\hline
\\
\textit{Actors:} & Users of sQuire.\\
\\
\textit{Goals:} & Delete an existing project. \\
\\
\textit{Pre-conditions:} & The user has the appropriate permissions to delete project. 
\\
\textit{Summary:} & A user deletes a project from the project workspace.\\
\\
\textit{Related use cases:} & Create a project. \\
\\
\textit{Steps:} & \begin{enumerate}
 \item The user selects a project.
 \item The user clicks on the "Delete project" button. 
 \item A dialog is displayed. 
 \item User select "delete" to delete the project. 
 \end{enumerate}\\
 \\
 \textit{Alternatives:} & User may choose not to delete the project in the confirmation display.\\
 \\
 \textit{Post-conditions:} & None. \\
 \\
\hline
\end{tabular}

\subsection{Export Project(Created by knic1468, Revised by bolt1003)}
\begin{tabular}{ p{2cm} p{12cm} }
\hline
\\
\textit{Actors:} & User of sQuire.\\
\\
\textit{Goals:} & Export a workspace to a local file. \\
\\
\textit{Pre-conditions:} & The user needs permission to export the project. 
\\
\textit{Summary:} & User saves a file containing the project settings and files to a local machine. \\
\\
\textit{Related use cases:} & Importing a project, Creating a new project. \\
\\
\textit{Steps:} & \begin{enumerate}
 \item The user clicks on the "Export File" button. 
 \item System prompts the user to select a location and filename. 
 \item User selects a file location.
 \item The user enters a file name.
 \item The user selects "export". 
 \end{enumerate}\\
 \\
 \textit{Alternatives:} & The user cancels the export, The system prompts that a file already exists with the same name.\\
 \\
 \textit{Post-conditions:} & None. \\
 \\
\hline
\end{tabular}

\subsection{Accept Invite to Project (Created by carl7595, Revised by bolt1003)}
\begin{tabular}{ p{2cm} p{12cm} }   
 \hline
 \\
 \textit{Actors:} & User who received the invite. \\
 \\
 \textit{Goals:} & Gain access to a Project. \\
 \\
 \textit{Pre-conditions:} & User has a valid email address. \\
 \\
 \textit{Summary:} & Access is granted to a project using an invitation email. \\ 
 \\
 \textit{Related use cases:} & Create an account. \\
 \\
 \textit{Steps:} & \begin{enumerate}
  \item Invitee clicks on the link received by email.
	 \item The link opens in a browser.
	 \item Dialog appear welcoming them to the project.
	 \item The project is added to their Projects list.
	\end{enumerate} \\
 \\
 \textit{Alternatives:} & The user ignores the invite. \\
 \textit{Post-conditions:} & Email link is deactivated. \\
 \\
\hline
\end{tabular}

\subsection{Remove User from Project (Created by carl7595, Revised by bolt1003)}
\begin{tabular}{ p{2cm} p{12cm} }   
 \hline
 \\
 \textit{Actors:} & User of sQuire \\
 \\
 \textit{Goals:} & Revoke access to the Project for a single or multuple users. \\
 \\
 \textit{Pre-conditions:} & The user has permission to edit the Project access list. \\
 \\
 \textit{Summary:} & One or more user accounts are removed from the access list for a project. \\ 
 \\
 \textit{Related use cases:} & Add users to a project.  \\ 
 \\
 \textit{Steps:} & \begin{enumerate}
  \item The user selects the access list for the project.
	 \item The user selects an account. 
	 \item The user selects "Remove from Project".
	 \item The user is prompted for confirmation
	 \item The user selects 'Yes'.
 \end{enumerate} \\
 \\
 \textit{Alternatives:} & The user selects 'No' and the access list is not modified. \\
 \\
 \textit{Post-conditions:} &
    \begin{itemize}
	 \item The user that was removed is notified of the change.
	 \item The user is prevented from accessing files.
    \end{itemize}\\
 \\
\hline
\end{tabular}

\subsection{Edit Project Permissions (Created by benz5834, Revised by bolt1003)}
\begin{tabular}{ p{2cm} p{12cm} }
 \hline
 \\
 \textit{Actors:} & User of sQuire \\ 
 \\
 \textit{Goals:} & Edit the permissions for a project \\
 \\
 \textit{Pre-conditions:} & The user is logged in. \\
 \\
 \textit{Summary:} & User opens up the settings menu and navigates to permissions, adds (or removes) users individual access rights to the project.  \\ 
 \\
 \textit{Related use cases:} & Add user to project, Remove user from project. \\ 
 \\
 \textit{Steps:} & \begin{enumerate}
  \item The user selects a project.
  \item The user selects settings.
  \item The user selects permissions.
  \item The user selects user from list of users.
  \item The user adds read or write permissions to user.
  \item The user selects save to save changes.
  \item The user exits settings.
 \end{enumerate} \\
 \\
 \textit{Alternatives:} & User can remove read or write permission instead in step 6. User can discard changes instead in step 7. \\
 \\
 \textit{Post-conditions:} & None. \\
 \\
\hline
\end{tabular}


\section{Settings - Preferences and Profile (brec9824)}
\subsection{View A User's Profile (brec9824)}
\begin{tabular}{ p{2cm} p{12cm} }
 \hline
 \\
 \textit{Actors:} & User of sQuire. \\ 
 \\
 \textit{Goals:} & User views a profile page. \\
 \\
 \textit{Pre-conditions:} & \begin{enumerate}
  \item The user is logged in.
 \end{enumerate} \\
 \\
 \textit{Summary:} & User clicks on their username or another user's name and a goes to a new page with the selected user's profile page.\\ 
 \\
 \textit{Related use cases:} & Modify Profile Info. \\ 
 \\
 \textit{Steps:} & \begin{enumerate}
  \item The user clicks on their username or another user's name.
  \item The user's system sends a request to the main sQuire system for the selected users profile information.
  \item sQuire system approves the request and sends the selected user's full profile info.
  \item The user's system loads a new page displaying the selected user's full profile info.
 \end{enumerate} \\
 \\
 \textit{Alternatives:} & In step 3 sQuire approves the request but because of the selected user's privacy settings only partial profile info is sent to the user. \\
 \\
 \textit{Post-conditions:} & None. \\
 \\
\hline
\end{tabular}

\subsection{Modify Profile Info (brec9824)}
\begin{tabular}{ p{2cm} p{12cm} }
 \hline
 \\
 \textit{Actors:} & User of sQuire. \\ 
 \\
 \textit{Goals:} & User updates their profile info including project preferences. \\
 \\
 \textit{Pre-conditions:} & \begin{enumerate}
  \item The user is logged in.
  \item The user is at their profile page.
 \end{enumerate} \\
 \\
 \textit{Summary:} & User clicks on the edit button, modifies their info, clicks save and their info gets saved.\\ 
 \\
 \textit{Related use cases:} & Modify Profile Privacy. \\ 
 \\
 \textit{Steps:} & \begin{enumerate}
  \item The user clicks the edit button.
  \item The user's system sends a request to the sQuire system.
  \item sQuire system recieves the request and verifies the user's credentials.
  \item The user's system loads a new page displaying the user's profile but with editable text boxes.
  \item The user edits their desired info.
  \item The user clicks the save button and the user's system sends the updated info to the sQuire system.
  \item sQuire system recieves the new data, verifies the data meets predefined requirements and aprroves the update.
  \item User is returned to their profile page as before with their updated info.
 \end{enumerate} \\
 \\
 \textit{Alternatives:} & \begin{enumerate} 
  \item In step 3 the sQuire system denies the request because the user was idle to long and is not logged in anymore. 
  \item In step 7 the sQuire system denies the user's request to update their profile: 1. the user's email was invalid 
  2. the user's password didn't meet the security requirements 3. the user used ineligible words or phrases. User is notified of the denial and is returned to step 5.
 \end{enumerate} \\
 \\
 \textit{Post-conditions:} & None. \\
 \\
\hline
\end{tabular}

\subsection{Modify Profile Privacy (brec9824)}
\begin{tabular}{ p{2cm} p{12cm} }
 \hline
 \\
 \textit{Actors:} & User of sQuire. \\ 
 \\
 \textit{Goals:} & User updates their profile info. \\
 \\
 \textit{Pre-conditions:} & \begin{enumerate}
  \item The user is logged in.
  \item The user is at their profile page.
 \end{enumerate} \\
 \\
 \textit{Summary:} & User clicks on the privacy level checkbox, clicks save and their new privacy level is saved.\\ 
 \\
 \textit{Related use cases:} & Modify Profile Info. \\ 
 \\
 \textit{Steps:} & \begin{enumerate}
  \item The user clicks the appropriate privacy checkbox next to the data they wont to change the privacy of.
  \item The user clicks the save button.
  \item sQuire system recieves the request and verifies the user's credentials.
  \item sQuire system recieves the updated privacy settings and aprroves the update.
  \item User is returned to their profile page as before with their updated info.
 \end{enumerate} \\
 \\
 \textit{Alternatives:} & \begin{enumerate} 
  \item In step 3 the sQuire system denies the request because the user was idle to long and is not logged in anymore.
 \end{enumerate} \\
 \\
 \textit{Post-conditions:} & None. \\
 \\
\hline
\end{tabular}

\subsection{Delete Account (brec9824)}
\begin{tabular}{ p{2cm} p{12cm} }
 \hline
 \\
 \textit{Actors:} & User of sQuire. \\ 
 \\
 \textit{Goals:} & Delete the user's account. \\
 \\
 \textit{Pre-conditions:} & \begin{enumerate}
  \item The user is logged in.
  \item The user is at their profile page.
 \end{enumerate} \\
 \\
 \textit{Summary:} & User clicks on the delete account button, then confirms their choice and their account is deleted from sQuire servers after a set amount of time.\\ 
 \\
 \textit{Related use cases:} & Modify Profile Info. \\ 
 \\
 \textit{Steps:} & \begin{enumerate}
  \item The user clicks the delete account button.
  \item System covers the room window with a new window that is dark and nearly transparent.(Gives the appearance that the page is dimmed) 
  \item System opens a pop-up window that contains a confirm button, a cancel button, and text that asks the user if they are sure and notifies them that this action is permanent.
  \item The user clicks the submit button.
  \item System closes the pop-up windows and the dim window in the background.
  \item System kicks the user from their account and adds the account to a list for future deletions.
  \item User is returned to sQuire's home page.
 \end{enumerate} \\
 \\
 \textit{Alternatives:} & \begin{enumerate} 
  \item If the user in step 4 clicks cancel or clicks out of the pop-up window and onto the dim window in the background. The dim window created in step 2 and the pop-up window in step 3 closes and action is taken.
 \end{enumerate} \\
 \\
 \textit{Post-conditions:} & None. \\
 \\
\hline
\end{tabular}

\subsection{Quick Jump To Projects (brec9824)}
\begin{tabular}{ p{2cm} p{12cm} }
 \hline
 \\
 \textit{Actors:} & User of sQuire. \\ 
 \\
 \textit{Goals:} & Go to a projects page that is listed in a user's profile. \\
 \\
 \textit{Pre-conditions:} & \begin{enumerate}
  \item The user is logged in.
  \item The user is at a user's profile page.
 \end{enumerate} \\
 \\
 \textit{Summary:} & User clicks on the appropriate project name and then is redirected to the projects home page.\\ 
 \\
 \textit{Related use cases:} & None. \\ 
 \\
 \textit{Steps:} & \begin{enumerate}
  \item The user searches through the projects listed in the user's profile page and clicks the project name they would like to go to.
  \item System recieves the request and searches for the specified project in the project data base.
  \item System finds the project and sends the redirect info.
  \item The user is then redirected to the given projects home page screen.
 \end{enumerate} \\
 \\
 \textit{Alternatives:} & \begin{enumerate} 
  \item None.
 \end{enumerate} \\
 \\
 \textit{Post-conditions:} & None. \\
 \\
\hline
\end{tabular}


\section{Project User Management (boss2849)}
\subsection{Modify read/write access (boss2849)}
\begin{tabular}{ p{2cm} p{12cm} }
 \hline
 \\
 \textit{Actors:} & User \\ 
 \\
 \textit{Goals:} & Modify a user\'s permissions. \\
 \\
 \textit{Pre-conditions:} & User is signed in and holds Admin rights for the currently selected Project\\
 \\
 \textit{Summary:} & User modifies another User\'s read/write permissions to portions of the project. \\
 \\
 \textit{Related use cases:} & None. \\ 
 \\
 \textit{Steps:} & \begin{enumerate}
  \item User clicks Permissions Management
  \item System displays permissions management window
  \item User selects a file, multiple files, directory or entirety of project and grants/revokes read/write access
  \item System modifes the target User\'s permissions and notifies them.
 \end{enumerate} \\
 \\
 \textit{Alternatives:} & 3. User selects cancel, System discards changes. \\
 \\
 \textit{Post-conditions:} & None. \\
 \\
\hline
\end{tabular}

\subsection{Remove User  (boss2849)}
\begin{tabular}{ p{2cm} p{12cm} }
 \hline
 \\
 \textit{Actors:} & User \\ 
 \\
 \textit{Goals:} & Remove a user from project. \\
 \\
 \textit{Pre-conditions:} & User is signed in, in project with Admin rights, and is on User Management page\\
 \\
 \textit{Summary:} & User removes a selected user from the Project \\ 
 \\
 \textit{Related use cases:} & Invite User, Modify Read/Write Access \\ 
 \\
 \textit{Steps:} & \begin{enumerate}
  \item User clicks Remove User button.
  \item System displays list of active users for project.
  \item User selects one or more other users from the list and presses Remove.
  \item System prompts User for verification.
  \item User presses Confirm.
  \item System removes the selected users from the project.
  \item System revokes read and write access from the selected users.
  \item System notifies selected users that they have been removed from the project.
 \end{enumerate} \\
 \\
 \textit{Alternatives:} & User presses Cancel in steps 3 or 5, System returns user to User Management page \\
 \\
 \textit{Post-conditions:} & None. \\
 \\
\hline
\end{tabular}

\subsection{Invite User to Project  (boss2849)}
\begin{tabular}{ p{2cm} p{12cm} }
 \hline
 \\
 \textit{Actors:} & User \\ 
 \\
 \textit{Goals:} & Invite user(s) to project \\
 \\
 \textit{Pre-conditions:} & User is signed in, in project with Admin rights, and is on User Management page \\
 \\
 \textit{Summary:} & User invites user(s) to the current project. \\ 
 \\
 \textit{Related use cases:} & Remove User, Join Project \\ 
 \\
 \textit{Steps:} & \begin{enumerate}
  \item User clicks Invite Users button
  \item System prompts user to enter username(s)/email(s)
  \item User enters username(s)/email(s) of the user(s) to invite and presses Ok.
  \item System looks up the specified user(s) and notifies them of invitation to the Project
 \end{enumerate} \\
 \\
 \textit{Alternatives:} & \begin{enumerate}
  \item User presses cancel in step 3, System returns User to User Management page
  \item In step 4, username(s)/email(s) don\'t match any users, System notifies User of failed invitiations.
 \end{enumerate} 
 \\
 \textit{Post-conditions:} & None. \\
 \\
\hline
\end{tabular}

\subsection{Promote User to Admin  (boss2849)}
\begin{tabular}{ p{2cm} p{12cm} }
 \hline
 \\
 \textit{Actors:} & User \\ 
 \\
 \textit{Goals:} & Promote a specified User to Admin \\
 \\
 \textit{Pre-conditions:} & User is signed in, in project with Admin rights, and is on User Management page \\
 \\
 \textit{Summary:} & User selects another User to be given Admin rights for the project. \\ 
 \\
 \textit{Related use cases:} & Demote Admin \\ 
 \\
 \textit{Steps:} & \begin{enumerate}
  \item User selects Promote to Admin.
  \item System displays a list of non-Admin active users.
  \item User selects user(s) and presses Submit.
  \item System prompts user for confirmation.
  \item User selects Confirm.
  \item System grants Admin permissions to the selected user(s).
 \end{enumerate} \\
 \\
 \textit{Alternatives:} & User presses cancel in steps 3 or 5, no action taken. \\
 \\
 \textit{Post-conditions:} & None. \\
 \\
\hline
\end{tabular}

\subsection{Demote Admin  (boss2849)}
\begin{tabular}{ p{2cm} p{12cm} }
 \hline
 \\
 \textit{Actors:} & User \\ 
 \\
 \textit{Goals:} & Demote Admin to user \\
 \\
 \textit{Pre-conditions:} & User is signed in, in project with Admin rights, and is on User Management page \\
 \\
 \textit{Summary:} & User demotes selected Admins to normal Users for the project. \\ 
 \\
 \textit{Related use cases:} & Promote User to Admin \\ 
 \\
 \textit{Steps:} & \begin{enumerate}
  \item User selects Demote Admin
  \item System displays list of Admins
  \item User selects Admin(s) to demote and presses Submit.
  \item System prompts User for confirmation.
  \item User presses Confirm.
  \item System revokes Admin rights from selected User(s)
 \end{enumerate} \\
 \\
 \textit{Alternatives:} & \begin{enumerate}
  \item User presses cancel in steps 3 or 5, no action taken
  \item User attempts to demote Admin that is the Owner of the project, System rejects request and notifies User.
 \end{enumerate}
 \\
 \textit{Post-conditions:} & None. \\
 \\
\hline
\end{tabular}

\subsection{Block User (boss2849)}
\begin{tabular}{ p{2cm} p{12cm} }
 \hline
 \\
 \textit{Actors:} & User \\ 
 \\
 \textit{Goals:} & Block a user from the project \\
 \\
 \textit{Pre-conditions:} & User is signed in, in project with Admin rights, and is on User Management page \\
 \\
 \textit{Summary:} & User blocks a user from the project, making them unable to view the project. \\ 
 \\
 \textit{Related use cases:} & Demote Admin \\ 
 \\
 \textit{Steps:} & \begin{enumerate}
  \item User clicks Block User.
  \item System displays a list of active users.
  \item User selects other user(s) to block and presses Submit.
  \item System prompts User for confirmation.
  \item User presses Confirm.
  \item System blocks selected user(s) from the project, revoking read/write access, and revoking Admin status as necessary.
 \end{enumerate} \\
 \\
 \textit{Alternatives:} & User presses cancel in steps 3 or 5.
 \\
 \textit{Post-conditions:} & None. \\
 \\
\hline
\end{tabular}

\section{Compile (boss2849)}
\begin{tabular}{ p{2cm} p{12cm} }
 \hline
 \\
 \textit{Actors:} & User \\ 
 \\
 \textit{Goals:} & Compile source \\
 \\
 \textit{Pre-conditions:} & User is logged in and viewing project. \\
 \\
 \textit{Summary:} & User requests that the code be compiled, the server compiles the code. \\
 \\
 \textit{Related use cases:} & Run, Compile To Jar \\ 
 \\
 \textit{Steps:} & \begin{enumerate}
   \item User selects ``Compile'' from ``Build'' dropdown menu for the current module.
   \item The Server receives the request to compile.
   \item The Server caches the current state of the project using the SnapshotManager and compiles it using the active CompilerPlugin.
   \item The Server returns the results of compilation to the User.
 \end{enumerate} \\
 \\
 \textit{Alternatives:} & In step 1, user chooses to compile entire project, including all sub modules. \\
 \\
 \textit{Post-conditions:} & None. \\
 \\
\hline
\end{tabular}

\section{Run (boss2849)}
\begin{tabular}{ p{2cm} p{12cm} }
 \hline
 \\
 \textit{Actors:} & User. \\ 
 \\
 \textit{Goals:} & Run the program. \\
 \\
 \textit{Pre-conditions:} & User is logged in and viewing a project. \\
 \\
 \textit{Summary:} & User chooses to run the program and the server compiles it or executes the last compilation result if no changes. \\
 \\
 \textit{Related use cases:} & Compile \\ 
 \\
 \textit{Steps:} & \begin{enumerate}
  \item User selects ``Run'' from the ``Build'' menu drop down.
  \item The Server receives the request to execute.
  \item The Server retrieves the most recent compilation from the SnapshotManager.
  \item The Server spawns a new window for the client that is the interface to the program.
 \end{enumerate} \\
 \\
 \textit{Alternatives:} & In step 3 the SnapshotManager has either an out of date compilation or no last compilation, the Server invokes the compiler to compile the project. \\
 \\
 \textit{Post-conditions:} & None. \\
 \\
\hline
\end{tabular}

\section{Package to Jar (boss2849)}
\begin{tabular}{ p{2cm} p{12cm} }
 \hline
 \\
 \textit{Actors:} & User \\ 
 \\
 \textit{Goals:} & Compile and package source to a jar\\
 \\
 \textit{Pre-conditions:} & User is signed in and viewing a project. \\
 \\
 \textit{Summary:} & User selects to build the project to a jar, the server outputs a jar on the project path. \\
 \\
 \textit{Related use cases:} & Compile \\ 
 \\
 \textit{Steps:} & \begin{enumerate}
  \item The user seclects ``Compile To Jar'' from ``Build'' dropdown menu.
  \item The Server receives the request to build a jar.
  \item The Server fetches the most recent compilation from the SnapshotManager.
  \item The Server packages the result of the last compilation to a jar and outputs it on the project path.
  \item The Server notifies the user of success.
 \end{enumerate} \\
 \\
 \textit{Alternatives:} & In step 3 the SnapshotManager has either an out of date compilation or no last compilation, the Server invokes the compiler to compile the project. In step 4 or 5, the compilation process fails and the Server notifies the user with the reason of failure. \\
 \\
 \textit{Post-conditions:} & None. \\
 \\
\hline
\end{tabular}

\section{Enable code freeze (boss2849)}
\begin{tabular}{ p{2cm} p{12cm} }
 \hline
 \\
 \textit{Actors:} & User \\ 
 \\
 \textit{Goals:} & Impose a code freeze on the project. \\
 \\
 \textit{Pre-conditions:} & User is signed in, viewing a project, and has admin rights.\\
 \\
 \textit{Summary:} & User places a code freeze on the project, preventing editing until undone. \\
 \\
 \textit{Related use cases:} & None. \\ 
 \\
 \textit{Steps:} & \begin{enumerate}
  \item The User selects ``Code Freeze'' from the dropdown menu.
  \item The Server receives the request for code freeze.
  \item The Server restricts all editing of project files.
 \end{enumerate} \\
 \\
 \textit{Alternatives:} & None. \\
 \\
 \textit{Post-conditions:} & None. \\
 \\
\hline
\end{tabular}

\end{document}